\section{Introduction}
\label{sec:CODAEMMA}

Les simulations présentées jusqu'ici ont toutes une taille de 8/h cMpc.
Cette taille présente l'avantage de pouvoir réaliser un grand nombre de simulations facilement sans que cela coute cher en temps de calcul.
Par contre il est admit que cette taille est trop faible pour étudier la réionisation dans son ensemble.
Les séries de simulation présentées ont pour vocation de calibrer et d'améliorer notre compréhension une simulation plus ambitieuse que je vais présenter dans ce chapitre.



%Lettre donc partie courte.

Cette partie repose sur une lettre qui s'inscrit dans une stratégie de travail a long terme.
Elle a pour vocation de présenter la simulation "CODA II EMMA" ainsi que les premiers résultats obtenus.
Ces résultats utilise deux simulation pour faire le lien entre la période de réionisation et l’époque actuelle.
Cette étude montre que les halos les plus massifs a z=0 sont réionisé plus tôt que le reste de l'Univers.

%Objectif connaitre le Z reio des halo en fonction de leur masse.
%Sur grosse simu donc beaucoup de stats

%\subsection{Présentation de la simulation CODA II EMMA}

\subsection{Conditions initiales}

Les conditions initiales ont été générée par la collaboration CLUES (Constrained Local UniversE Simulations).
L'objectif est de retrouver dans la simulation des structures aillant des caractéristiques proches de ce qui est observé dans l'univers local.
On cherchera par exemple a obtenir un couple Andromède - Voie Lacté avec des masses, distances et vitesses relative en accord avec les contraintes actuelles.

%cosmo

Le volume de $\left( 64/h cMpc \right) ^3$ est échantillonné par $2048^3$ particules de matière noire.
Ce qui mêne a une résolution en masse $3.4 \cdot 10^6 M_\odot$ et une résolution spatiale de 46 ckpc sur la grille coarse.
Ces paramètres permettent d'explorer la gamme de masse de halos compris entre $10^8 M_\odot$ et  $10^{13}M_\odot$.

Ces conditions initiales sont une version basse résolution de celles de la simulation CODA I%TODO ref
réalisé avec le code RAMSES CUDATON %TODO ref
initialement en $4096^3$.
La perte de un niveau de résolution est compensé par l'utilisation de l'AMR de EMMA.

L'objectif est de pouvoir faire une comparaison directe entre ces deux simulations (CODA I RAMSES CUDATON vs CODA II EMMA) dans un avenir proche.

%différences avec CODA I RAMSES CUDATON et future comparaisons
%8x moins résolue en masse 
%Mais mieux résolue en dx


\subsection{Présentation des simulations}

A partir de ces conditions initiales plusieurs simulations ont été réalisées:

\begin{itemize}
\item Une simulation matière noire pure réalisée avec Gadget %TODO ref
exécutée jusqu'à redshift z=0.
Cette première a pour objectif de suivre l'évolution des structures.

\item Et une simulation en RHD entièrement couplé avec EMMA poussée jusqu'à z=6 à la fin de la simulation.
Cette dernière a été exécutée sur 32768 cœurs CPU et 4096 GPU du calculateur TITAN.
Elle a pour but d'avoir une représentation complète de l'époque de réionisation.
\end{itemize}



\section{projection a z=0}

L'objectif est de créer un lien entre ces deux simulations.
La simulation matière noire a servi a créer un arbre de fusion (merger tree) des halos, permettant de suivre les histoires de formation des halos du début de la simulation jusqu'à nos jours.

En considérant que le couplage entre la matière noire et les baryons et faible, la comparaison entre les catalogues de halos des deux simulations pris a la même époque peux être réalisé directement.


les 2 méthodes :

- Centre de masse pris dans le merger tree
- moyenne du t des particules



\section{Résultats}



