\chapter{Le modele standard de la cosmologie moderne} \label{ch:introduction_physique}

Cette thèse s'inscrit dans le cadre du modèle standard de la cosmologie.
Ce modèle, aussi appelé modèle du Big Bang, considère un univers en expansion et composé essentiellement de matière noire froide $\Lambda$ Cold Dark Matter ($\Lambda$CDM).
Ce chapitre a pour objectif de présenter les grandes lignes de ce modèle.


\section{Émergence de l'idée d'un Univers non statique}

%découverte des galaxies\\
%découverte de l'expansion de l'univers

\subsection{Cadre théorique}

L'idée d'un univers non statique a pris forme dans le début du XIX eme siècle suite a deux événement majeurs.
D'un coté l'élaboration de la théorie de la relativité générale d'Einstein en 1915 a permis la mise en place d'un cadre théorique propice.
%La notion d'un univers non statique a ete introduite par Einstein en 1917 en rapport avec ses travaux sur le relativité générale. 
D'un autre coté, plus d'une décennie plus tard les observation de Hubble entre 1923 et 1929.

\subsection{Les observations de Hubble}

Hubble observe deux phénomènes qui vont mené a une redéfinition de notre notion de cosmologie.
Il bénéficie a l'époque de l'accès au plus puissant télescope du monde (le télescope Hooker du mont Wilson)

Cet instrument lui a permis, dans une publication de 1926, d'observer que des nébuleuses situées hors de notre galaxie présente des caractéristiques identiques a des système d'étoile \citep{1926ApJ....63..236H}.
Il en déduit qu'il existe d'autre galaxie que notre voie lactée.

Avec les moyens observationnels dont nous disposons aujourd'hui l’existence d'un grand nombre de galaxies en dehors de notre voie lactée est bien établie.
La figure \ref{fig:hubbl_deep_field} présente le "Hubble Ultra Deep Field" \citep{1538-3881-132-5-1729} une image prise en 2014, montrant un grand nombre de galaxies dans une portion réduite du ciel.
On estime aujourd'hui le nombre de galaxies à $\approx 2 \cdot 10^{11}$.


\begin{figure}[bth]
        \includegraphics[width=.9\linewidth]{img/01/hudf.jpeg} 
        \caption{Hubble Ultra Deep Field} 
        %2014.
        %Image NASA.
		En 1926 E. Hubble observe pour la première fois qu'il existe d'autre galaxies.
		Leur nombre est estimé aujourd'hui à $\approx 2 \cdot 10^{11}$.
 		\label{fig:hubbl_deep_field}
\end{figure}

\subsection{La loi de Hubble}
Hubble observe également une relation entre la distance de ces nouvelles galaxies et leurs spectre lumineux \citep{1929CoMtW...3...23H}.
Plus ces galaxies sont éloignées de l'observateur, plus leurs spectre est décalé vers le rouge.

Ce décalage vers le rouge -- ou redshift en anglais -- est une notion qui sera beaucoup utilisée par la suite.
%En cosmologie nous referons a l'age de l'Univers autant en temps qu'en redshift.
Nous verrons dans la prochaine section qu'il existe un lien direct entre age de l'univers et redshift.

Il interpréta ce décalage comme un effet Doppler et montra que ces galaxies s'éloignent de l'observateur avec une vitesse radiale directement proportionnelle a leur distance.
Cette corrélation entre la distance et le décalage vers le rouge observé est aujourd'hui très bien établie observationnellement est présenté sur la Fig. \ref{fig:hubble_law} extraite de \citep{2015PNAS..112.3173B}.

\begin{figure}[bth]
        \includegraphics[width=.9\linewidth]{img/01/hubble_law.jpg} 
        \caption{Loi de Hubble a partir d'observation actuelles. 
%http://www.pnas.org/content/112/11/3173/F2.expansion.html
%        Image ESO
        }
 		\label{fig:hubble_law}
\end{figure}

Cette relation entre la distance des galaxies $D$ et leurs vitesse radial d'éloignement $V$ est aujourd'hui appelée loi de Hubble et peux être résumé par :

\begin{equation}
V = H_0 D,
\end{equation}
ou $H_0 = 67 \mathrm{ \left[ km.s^{-1}.Mpc^{-1} \right ] }$ est la constante de Hubble aujourd'hui \citep{planck_collaboration_planck_2016}.

Conventionnellement, le sous script $0$ désigne le fait que la valeur de la variable qui lui est associé est celle prise aujourd'hui.
Cette distinction est nécessaire car la valeur de la constante de Hubble n'est pas constante dans le temps.
D'une manière plus générale la constante de Hubble s'exprime sous la forme
\begin{equation}
H=\frac{1}{a} \frac{da}{dt} = \frac{\dot{a}}{a}
\end{equation}

ou $a = a_{(t)}$ est le facteur d'expansion.
Sa valeur représente la taille de l'Univers relativement a sa taille actuelle.
Par définition $a_0 = 1$.

Le redshift, noté $z$, s'exprime en fonction du facteur d'expansion de la manière suivante:

\begin{equation}
z= \frac{1}{a}-1
\end{equation}

\subsection{Équation d'Einstein}

Les observations de Hubble on permis de confirmer ce qui avait été pressentis par Einstein plus d'une décennie plus tôt, lorsqu'il introduisit le concept de constante cosmologique ($\Lambda$). 

Einstein publie en 1916 sa fameuse théorie de la relativité générale \citep{1916AnP...354..769E}.
Il y introduit sa célèbre equation de champ décrivant le lien entre densité d'énergie et déformation de l'espace-temps:

%Equation d'Einstein 
%http://cdsads.u-strasbg.fr/abs/1915SPAW.......844E
\begin{equation}
R_{\mu\nu} = \frac{1}{2} g_{\mu\nu}R + \Lambda g_{\mu\nu}  = \kappa T_{\mu\nu}.
\label{eq:einstein}
\end{equation} 

Dans cette equation apparaît le terme $\Lambda$ (de $\Lambda$CDM), appelée \textit{constante cosmologique}.
Cette constante représente mathématiquement le fait que l'espace-temps peut être en expansion ou en contraction.


\subsection{Équations de Friedmann}
\label{sec:friedman}
%Metrique de Friedmann-Lemaître-Robertson-Walker (FLRW)


Quelques années plus tard, Alexandre Friedmann entreprend de trouver des solutions exactes a l'équation d'Einstein en l'appliquant a un Univers homogène et isotrope \citep{1922ZPhy...10..377F}.
Il arrive a ce système d'équations indépendantes permettant de modéliser l'Univers dans son ensemble :

%Recriture de l equation d'Einstein en considerant un univers homogene et isotrope.
% 
%Alexandre Friedmann Über die Krümmung des Raumes, Zeitschrift für Physik 10, 377-386 (1922). 
%Première écriture des équations de Friedmann, dans le cas d'une coubure spatiale positive. http://cdsads.u-strasbg.fr/abs/1922ZPhy...10..377F 
%
%(de) Alexandre Friedmann, Über die Möglichkeit einer Welt mit konstanter negativer Krümmung des Raumes, Zeitschrift für Physik 21 326–332 (1924). 
%Écriture des équations de Friedmann dans le cas d'une courbure spatiale négative. 
% 
%univers de Friedmann-Lemaître-Robertson-Walker  
%http://dictionnaire.sensagent.leparisien.fr/%C3%89quations%20de%20Friedmann/fr-fr/
 
%Équations de Friedmann : 

\begin{equation}
3 \left( \frac{H^2}{c^2} +\frac{K}{a^2} \right) = \frac{8 \pi G }{c^2} \rho
\label{eq:friedman1}
\end{equation}

\begin{equation}
-2 \frac{ \dot{H}}{c^2} -3 \frac{H^2}{c^2} -\frac{K}{a^2} = \frac{8 \pi G }{c^4 P}
\label{eq:friedman2}
\end{equation}

L'équation \ref{eq:friedman1} relie le taux d'expansion H, la courbure spatiale K et le facteur d'échelle a à la densité d'énergie $\rho$.
L'équation \ref{eq:friedman2} relie la pression P à la dérivée temporelle du taux d'expansion.
 
  
En manipulant l'équation \ref{eq:friedman1} il est possible de montrer que :
% cf 
% http://dictionnaire.sensagent.leparisien.fr/%C3%89quations%20de%20Friedmann/fr-fr/
% https://en.wikipedia.org/wiki/Friedmann_equations
% pour la demonstration

%TODO introduire les differents fluides cosmologiques

\begin{equation}
\frac{\dot{a}}{a} = H_0 \sqrt{ \Omega_{r,0} a^{-4} +  \Omega_{M,0} a^{-3} + \Omega_{K,0}a^{-2} + \Omega_{\Lambda,0}  } 
\label{eq:scale_t}
\end{equation}


\begin{figure}[bth]
        \includegraphics[width=.9\linewidth]{img/01/dark4.jpg} 
        \caption{Importance respective des differentes densité d'énergie en fonction de l'age de l'Univers (et donc de sa taille). Image extraite de \citep{2005univ.book.....F}
        }
 		\label{fig:cosmoparamt}
\end{figure}


Ou les paramètres $\Omega_{i,0}$ représente la densité d'énergie associé aux différent constituant, 
\begin{equation}
 \Omega_{i,0} = \frac{\rho_{i,0}}{\rho_{c,0}},
 \end{equation}

en fonction de la densité critique de l'Univers:

\begin{equation}
\rho_{c,0} = \frac{3H_0^2}{8\pi G}
 \end{equation}
Cette valeur est déterminée avec l'équation \ref{eq:friedman1}, en considérant un univers statique ($\Lambda=0$) et plat ($\rho_{K,0}=0$).
Nous verront dans la partie dédiée a la déterminations des paramètres cosmologique (cf Sec \ref{cosmoparam}) que les mesures actuelles sont en faveur d'un Univers plat (sans courbure).
Ce qui s'exprime par :

\begin{equation}
\Omega_{K,0} = 0,
\end{equation}

il s'en suis que :
\begin{equation}
\Omega_{\lambda,0} +  \Omega_{M,0} + \Omega_{r,0} =1 .
\end{equation}


\begin{itemize}

\item $\Omega_{M,0}$ est la densité d'énergie associée a la matière (matière noire que nous introduiront dans la suite et la matière baryonique)
Par simple effet de dilution, cette densité décroît avec le cube du facteur d'expansion.
Un Univers constitué exclusivement de cette énergie serait nommé "Univers poussière"
C'est le cas actuellement.

\item $\Omega_{r,0}$ représente la densité d'énergie radiative.
La densité de photon décroît avec le cube du facteur d'expansion, mais du fait de l'allongement de la longueur d'onde avec $a$, la densité d'énergie associée a la radiation décroit avec la puissance $4$ de $a$.
Un Univers constitué exclusivement de cette énergie serait nommé "Univers lumière".
Cette approximation est valable pour un Univers majoritairement remplis de matière relativiste.
C'était le cas avant L’émission du fond diffus cosmologique.


\item $\Omega_{\Lambda,0}$ est la densité d’énergie du vide.
Cette énergie est associée a la constante cosmologique et a l'expansion de l'Univers. 
Cette densité est constante dans le temps

\item $\Omega_{K,0}$ est densité de courbure spatiale ou la densité d'énergie associée a la courbure de l'espace.
Selon l’équation \ref{eq:einstein}, l'énergie peut courber l'espace, mais si l'espace possède une courbure intrinsèque, elle est équivalente a une densité d'énergie.
\begin{equation}
\Omega_{K,0} = 1 - \Omega_{r,0} - \Omega_{M,0} - \Omega_{\lambda,0} 
\end{equation}

\end{itemize}





\begin{figure}[bth]
        \includegraphics[width=.9\linewidth]{img/01/scale_t.jpg} 
        \caption{Taille de l'Univers en fonction du temps pour différents paramètres cosmologiques.
%https://map.gsfc.nasa.gov/universe/bb_concepts.html#Expansion
        }
 		\label{fig:scale_t}
\end{figure}

En intégrant l’équation \ref{eq:scale_t}, il est possible de déterminer l'évolution de la taille de l'Univers en fonction de la densité respectives de ses différents constituants.
La figure \ref{fig:scale_t} présente quelques évolutions possibles.
Toutes les courbes passe par le point (Now,1) car par définition $a_0 = 1$.
%TODO expliquer cette histoire de tangente
Toutes les courbes ont la même tangente au point (Now,1).
Nous observons ici que la valeur des paramètres cosmologique contraint l'age actuel de l'Univers (le point où la courbe passe l'axe horizontal).
La valeur des paramètres contraint aussi l'avenir de l'Univers.
Par exemple, la courbe orange représente un Univers dit fermé.
Ce type d'Univers est voué a s'effondrer sur lui même a cause de la gravité de la matière qu'il contient.
A l'inverse, la courbe rouge représente un Univers dit ouvert.
La gravitation n'est pas suffisante pour contrer l'expansion, et ce type d'Univers est voué a grandir indéfiniment.

%\cite{2003PhT....56d..53P}


\subsection{Énergie noire}

%Dans un Univers, ou seule la gravité agit a grandes échelles, l'expansion ne peux que décélérer.


Si nous avons vu que la taille de l'Univers évolue avec le temps, il reste la question de l’évolution de cette vitesse.
Si l'on considère que l'univers s’étend suite a une impulsion initiale (cf prochaine section), et que la gravité est la seule force qui agit a grande échelle, l'expansion devrait décélérer.
En effet comme la gravité est une force purement attractive, elle agis comme un ressort qui devrait ramener toute la masse a l'origine (scénario de la courbe orange sur la figure \ref{fig:scale_t}, ou la courbe retombe sur l'axe horizontal signifiant que la taille de l'univers tend a nouveau vers zero).

Or ce n'est pas ce qui est observé: l'accélération de l'expansion de l'univers a été découverte a la fin du XXème siècle par 2 équipes simultanément :
\begin{itemize}
\item  High-Z supernovae search team \citep{1998AJ....116.1009R}%http://cdsads.u-strasbg.fr/abs/1998AJ....116.1009R
\item  Supernova Cosmology Project \citep{1999ApJ...517..565P}% http://cdsads.u-strasbg.fr/abs/1999ApJ...517..565P
\end{itemize}
Les 2 ont obtenues le prix Nobel en 2011.

L'accélération de l'expansion de l'Univers pose un problème majeur.
Il est établis depuis les travaux de Newton que pour accélérer une masse une force est nécessaire.
Or nous ne disposons actuellement d'aucune théorie permettant d'expliquer cette force répulsive.
Il a fallu seulement quelques mois après cette découverte pour qu’apparaisse pour la première fois la mention d'énergie noire: \citep{1999PhRvD..60h1301H}.


%échelle gigaparsec\\
%Facteur d'expansion
%
%

\section{Le modèle du Big bang}


%le big bang\\
%l'inflation\\
%la nucléosynthèse\\
%le CMB\\
%la reionization


Une fois établis que les galaxies s'éloigne de nous dans toutes les directions, plusieurs constats s'imposent.

Premièrement, il est possible d'imaginer qu'en remontant le temps, elles devaient être plus proches les une des autres, et en poussant se constat a l’extrême, il fut imaginé qu'a un certain instant dans l'histoire de l'Univers, toute la matière devait être concentré en un point (toutes les courbes de la figure \ref{fig:scale_t} passe par une taille nulle dans le passé)
Cette singularité à été baptisé Big Bang et donne son nom au modèle cosmologique actuel.

En 1927, Georges Lemaire propose une théorie de l'atome primitif \cite{1927ASSB...47...49L} 


\subsection{Température de l'Univers}

Il existe un lien direct entre la taille de l'Univers et sa température.
Dans la conception actuelle de l'Univers, celui ci est par définition l'ensemble de ce qui existe, il ne lui est pas possible d'échanger de l’énergie avec l’extérieur et c'est donc un système isolé.
Or, la thermodynamique nous dit qu'il existe une relation entre le volume et la température d'un system adiabatique.
La température (la densité d'énergie interne) diminue au fur et a mesure que l'Univers se dilate.
Plus l'on se rapproche temporellement du BigBang, plus l'Univers est dense et chaud.


\subsection{Nucléosynthèse primordiale}
\label{sec:nucleosynthese_primordiale}
%Dans le modèle du Big Bang, le début de l'Univers est suivis par une phase d'expansion rapide appelé inflation.
Après le BigBang, l'Univers est composé d'une soupe de particules élémentaire (quark, gluon, électron) qui vont former en se refroidissant les premiers protons, neutrons et électrons.
Les protons et les neutrons vont a leurs tour s'assembler pour former les premiers noyaux atomiques.
La théorique de la nucléosynthèse primordiale a été présenté dans un article surnommé $\alpha \beta \gamma$ due aux initiales des noms de ses auteurs \citep{PhysRev.73.803}.

Cette théorie permets d'expliquer avec precision l'abondance observée des différents atomes présent dans l'Univers.

\begin{itemize}
\item 73,9\% D’Hydrogène
\item 24\% D’Hélium
\item Le reste (2.1\%) est constitué de l'intégralité des autres éléments du tableau périodique que nous nommerons "métaux"
\end{itemize}

\subsection{Refroidissement du plasma primordial}
A cette période, les noyaux atomiques sont découplé des électrons, on dit que l'Univers est dans un état ionisé.
Un plasma proche de celui que l'on peux trouver au centre d'une étoile.
Il faudra attendre $\approx 380 000$ ans pour que la température baisse suffisamment pour permettre l'apparition des premiers atomes neutres.
Cet période est appelée "l'époque de la recombinaison" et a donné lieu a l'émission du fond diffus cosmologique que nous développerons plus en détails dans la section \ref{sec:CMB}.
Durant cette transition, l'Univers a connu un changement d'état majeur puisque celui ci est passé d'un état globalement ionisé a un état globalement neutre.

Robert Herman et Ralph Alpher ont été les premiers a proposer l’existence du fond diffus cosmologique avant même sa découverte

revue sur la recombinaison :\ref{2009AN....330..657S}

\section{Le fond diffus cosmologique}
\label{sec:CMB}

Comme nous l'avons vu dans la section précédente, l’émission du fond diffus cosmologique, ou rayonnement de fond micro-ondes (pour Cosmic Microwave Background ou CMB) marque la transition entre deux état distincts de l'Univers.



\subsection{Surface de dernière diffusion}

Du au grand nombre d'électrons libres avant la recombinaison, la lumière était soumise a un grand nombre de diffusions.
Et a la manière de la surface d'une étoile, ou nous ne voyons que les photons qui ont pu s’échapper de celle ci et non ceux qui ont été émis en son centre, nous ne pouvons pas observer de lumière émise avant le CMB.

Le CMB étant la plus ancienne lumière que nous pouvons observer, il contient actuellement l'information que nous pouvons obtenir sur l'état de l'Univers lorsqu'il était le plus jeune possible.


\subsection{Observations}

La quête de l'état de l'Univers a ses début a commencé de manière fortuite en 1964 quand Penzias et Willson ont observé, lors de travaux sur un nouveaux type d’antennes radio, un signal radio inexpliqué.
Ce signal était constant et extrêmement homogène. 
%(l'anisotropie du CMB est remis en cause recement 
%http://www.ca-se-passe-la-haut.fr/2013/09/lanisotropie-du-fond-diffus-cosmologique.html)

Ils obtiennent le prix Nobel en 1978 pour la \cite{PenziasWilsonNobel}.


\subsection{Température}
Le cosmic Microwawe background se présente sous la forme du corps noir a une température de 2.73°K.
Fig. \ref{fig:cmb_thermal_spectrum}
T=2.73K

emis a z=1100 et a une température de $\approx 3000$ K.



\begin{figure}[htbp]
        \includegraphics[width=.95\linewidth]{img/01/Cmbr.pdf} 
        \caption{Spectre thermique du CMB vue par le satellite Cosmic Background Explorer (COBE). 
        Image Wikipédia}
 		\label{fig:cmb_thermal_spectrum}
\end{figure}




\subsection{Spectre de puissance}


L'observation de Penzias et Wilson constitue un argument de poids en faveur de la théorie du Big Bang et a été suivie d'une série de mission spatiale dans le but d'améliorer les mesures faites sur le CMB.

\begin{itemize}
\item 1989 - le satellite COsmic Background Explorer COBE 
\item 2001 - le satellite Wilkinson Microwave Anisotropy Probe WMAP
\item 2009 - le satellite Plank
\end{itemize}

John C. Mather et George F. Smoot ont conjointement obtenus le prix Nobel de physique en 2006 pour la \cite{CMBanisotropiesNobel} grâce aux observations réalisées par le satellite  (COBE).
Ce léger défaut d'uniformité (de l'ordre de $10^{-5}$ en relatif) nous renseigne sur L’état de l'univers au moment de son émission.
Par la suite, les satellites WMAP et Planck ont grandement amélioré la précision des mesures des anisotropies du fond diffus (Fig. \ref{fig:cmb}).

\begin{figure}[htbp]
        \includegraphics[height=.95\textheight]{img/01/CMB.jpeg} 
        \caption{Les fluctuations du CMB vues par le satellite Planck. 
        Image ESA}
 		\label{fig:cmb}
\end{figure}


On étudie les anisotropies du CMB en décomposant ces fluctuations en harmoniques sphériques:
Fig\,\ref{fig:harmoniques_spheriques}

decomposition en multipoles
%https://www.physicsforums.com/threads/can-someone-explain-angular-power-spectrum.309483/
\begin{equation}
 \frac{\Delta T(\theta,\phi)}{T} = \sum_{l>0} \sum_{m=-l}^l a_{lm} Y(\theta,\phi)_{lm}
\end{equation}

avec : 

\begin{equation}
a_{lm}= \int d\Omega(\theta,\phi) \Delta T (\theta,\phi) Y(\theta,\phi)_{lm}
\end{equation}

%\begin{figure}[bth]
%        \includegraphics[width=.95\linewidth]{img/01/harmoniques_spheriques.jpeg} 
%        \caption{
%        représentation des $Y(\theta,\phi)_{lm}$
% Pierre Brassard, université de Montréal 
%%Spectre thermique du CMB vue par le satellite Cosmic Background Explorer (COBE). 
%        Image Wikipédia}
% 		\label{fig:harmoniques_spheriques}
%\end{figure}


\begin{equation}
C_l = \frac{1}{2l+1} \sum_{m=-l}^l a_{lm} a_{lm}^*
\end{equation}


Et finalement, on obtient le spectre de puissance:

\begin{equation}
D_l = \frac{l (l+1) C_l }{2 \pi} 
\end{equation}

représenté Fig.\,\ref{fig:cmb_power_spectrum}

\begin{figure}[bth]
        \includegraphics[width=.95\linewidth]{img/01/CMB_power_spectrum.png} 
        \caption{Spectre de puissance des fluctuation du CMB.
        Image ESA}
 		\label{fig:cmb_power_spectrum}
\end{figure}





\section{Le contenu de l'univers}
\label{cosmoparam}

\begin{figure}[bth]
        \includegraphics[width=.95\linewidth]{img/01/cosmoparam.png} 
        \caption{Détermination des paramètres cosmologique a partir de différents observables. Figure extraite de \cite{2008ApJ...686..749K}}
 		\label{fig:cosmoparam}
\end{figure}

Fig. \ref{fig:cosmoparam}


Pour simuler l'univers, on a besoin de savoir ce qu'il contient. 
A partir du spectre de puissance, on peut déterminer les différentes composantes de l'univers (paramètres cosmologique).

univers infini, homogène, isotrope


%https://ned.ipac.caltech.edu/level5/Freedman2/Freed6.html


\begin{figure}[bth]
        \includegraphics[width=.95\linewidth]{img/01/table_planck.pdf} 
        \caption{Determination des parametres cosmologiques par la colaboration Planck.}
 		\label{fig:planck_parameters}
\end{figure}

\citep{planck_collaboration_planck_2016}

\subsection{Matière noire CDM}

echelle mega parsec\\
gouverne la gravité\\
non collisionnelle\\


\begin{figure}[bth]
        \includegraphics[width=.95\linewidth]{img/01/matter_power_spectrum.jpeg} 
        \caption{Spectre de puissance de distribution de la matière a grande échelle
        %http://adsabs.harvard.edu/cgi-bin/bib_query?2004PhRvD..69j3501T
        }
 		\label{fig:matter_power_spectrum}
\end{figure}



\subsection{Baryon}

echelle kilo parsec
collisionnelle
interagit avec la radiation
La matière visible

\subsection{Radiation}

quasiment notre seul source d'information sur l'univers (plus vrai depuis les ondes gravitationnelles)
essentielle pour la reionization
seulement E>13.6 eV

\subsection{bilan}

plot en camembert avec les différents constituants


nessecité d'introduire la matière noire

\section{dark matter}







A la suite de l'émission du fond diffus cosmologique, commence une période appelée "les ages sombres".
L'Univers est alors composé de gaz froid soumis principalement a deux forces : la gravité et l'expansion de l'Univers.
La compétition entre ces deux forces couplé a de très légères perturbations dans la densité de l''Univers ont menées a l'apparition des premières sur-densité qui ont permis l'apparition des premières étoiles.
Ces étoiles ont émis du rayonnement suffisamment énergétique pour arracher les électrons du gaz environnant.
L'Univers va alors subir un second changement d'état majeur, puisque le rayonnement des premières étoiles va de nouveau ioniser le gaz.
C'est l'époque de la reionization.




%perturbation lineaires:
%https://ned.ipac.caltech.edu/level5/Sept11/Norman/Norman2.html

