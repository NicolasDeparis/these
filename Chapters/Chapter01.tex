\chapter{Introduction au modèle physique }\label{ch:introduction}

les 3 piliers de l'astrophysique:

\begin{enumerate}
\item observation
\item théorie
\item simulation
\end{enumerate}


l'observation et la théorie sont les plus ancien piliers, les hommes ont toujours regardé le ciel, et ont toujours essayé de comprendre ce qu'ils observaient. La simulation est beaucoup plus récente et dépend grandement de la technologie.

lien avec la méthode scientifique de manière générale. observation, modélisation et test de la théorie or en astro on ne peut pas tester directement donc on simule.

\section{observation -> Hubble}

découverte des galaxies
découverte de l'expansion de l'univers


\section{théorie - lCDM}

le big bang
l'inflation
la nucléosynthèse
le CMB
la reionization


\section{observation -> le CMB}

Penzias et Willson
Corps noir parfait
surface de dernière diffusion
spectre de puissance

\section{Théorie-> le CMB et le contenu de l'univers}

Pour simuler l'univers, on a besoin de savoir ce qu'il contient. A partir du spectre de puissance, on peut déterminer les différentes composantes de l'univers (paramètres cosmologique).

univers infini, homogène, isotrope

\subsection{Energie noire}

echelle gigaparsec
Facteur d'expansion

\subsection{Matière noire}

echelle mega parsec
gouverne la gravité
non collisionnelle

\subsection{Baryon}

echelle kilo parsec
collisionnelle
interagit avec la radiation
La matière visible

\subsection{Radiation}

quasiment notre seul source d'information sur l'univers (plus vrai depuis les ondes gravitationnelles)
essentielle pour la reionization
seulement E>13.6 eV

\subsection{bilan}

plot en camembert avec les différents constituants

\section{Observation -> la reionization}

le manque d'observations

la difficulté des observations

les futures observations

Quelles sont les preuves de la réionisation?

spectre de quasar

polarisation du CMB

ligne 21 cm

fonction de luminosité UV

Epaisseur optique lyman alpha

Epaisseur optique Thomson

\section{Théorie -> La reionization}

réionisation et non rayonnisation!

Qu'est ce que c'est?

fin des âges sombres
apparition des première sources de rayonnement
Pourquoi étudier la réionisation

Dernier processus impactant l'ensemble de l'univers.
Importance pour le "missing satellite problem"

\subsection{les principales question en suspend de l'étude de la réionisation}

quand est ce arrivé?
quelles sont les sources? -> débat galaxies vs quasars
outlier dans l'épaisseur optique des quasars
Le groupe local ?


