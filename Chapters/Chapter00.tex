
\chapter*{Avant-Propos}

\begin{flushright}{\slshape    
	Une civilisation sans la science \\
	c'est aussi absurde qu'un poisson sans bicyclette.} \\ \medskip 
	--- Pierre Desproges
\end{flushright}

%\begin{itemize}
%\item la révolution industrielle
%\item migration vers les villes (50\% de la population mondiale depuis pas longtemps)
%\item déconnexion de la terre a cause du béton
%\item déconnexion du ciel a cause des éclairages publique
%\item étudier l'astrophysique est essentiel pour que l'Homme reste humble et considère sa place dans l'univers pour ne pas courir a sa perte.
%\end{itemize}

De nombreuses cosmologies ont vues le jours au fil des siècles et des peuples.
La question de la place de l'Humanité dans l'Univers a toujours été centrale et au cœur des préoccupations de toutes les civilisations.
%Toutes ont en commun 

Il fut un temps ou l'Homme n'avait d'autre choix que de cultiver les champs le jour et vivre dans l'obscurité la nuit.
Il existait un lien fort entre ce qui se passait sur terre (Humanité vient de Humus après tout) et dans le ciel.
Il ne pouvait que constater le ciel étoilé et se poser la question de la provenance des lueurs qui remplissent la voûte céleste.

L’absence de lumière a toujours créer une peur de l'inconnu, cette fameuse peur du noir.
Depuis le XIX ème siècle et la revolution industrielle, il est devenu possible de s'affranchir de la nuit.
Et nous ne nous en sommes pas privé.

Nous vivons dans un monde ou plus de la moitié de la population mondiale vie dans des villes ou l'obscurité n'a pas sa place.
L’éclairage publique agis comme un écran qui nous bloque la vue du ciel, et par la même occasion inhibe cette sensation de vertige que l'on peu ressentir en le regardant.

Cette migration urbaine a aussi pour conséquence de nous couper du lien a la terre
Les villes sont presque intégralement étanchéifier par du béton.

L'étude de la cosmologie et sa vulgarisation est une nécessité si l'on veux retrouver une certaine humilité vis a vis de notre seul et unique lieu de vie.




\begin{flushright}{\slshape    
	You devellop an instant global consciousess, \\
	a people orientation,\\
	an intense dissatisfaction with the state of the world,\\
	and a compulsion to do something about it.\\
	From out there on the moon, internationnal politic look so petty. \\
	You want to grab a politician by the scuff of the neck\\
	and drag him a quarter of a million mile out and say: \\
	'Look at that you son of a bitch'\\ \medskip 
	--- Edgar Mitchell Appolo 14 astronaut }
\end{flushright}