\chapter{Introduction au modèle numérique}\label{ch:introduction}

Les echelles de temps sont radicalement opposées entre la cosmo qui considère les temps les plus long de l'univers et les progrès informatiques qui vont a une vitesse exponentielle. il faut considérer les simulations comme éphémère.

Comment modéliser la reionization?

\section{Les différents types de codes}

( introduction au différentes représentations (particules/grille) nécessaire pour la suite )

historique
avantage inconvénient AMR vs SPH
introduction de la grille et de la méthode AMR

\section{Gestion de la grille}

(nécessaire d'être positionné ici car la structure en arbre conditionne plusieur choix par la suite)

Oct tree
gestion du raffinement
cell linked list
Energie noire

\section{Système d'unités supercomobiles}
le pas de temps
Matière noire

\section{génération des conditions initiales}

méthode
gaussian random noise
théorie des perturbation linéaire
lien avec le spectre de puissance
MUSIC et GRAPHIC
limite la résolution min et max (min en masse et max en espace)
théorie des perturbation linéaire

\section{approximation de zeldovich}
perte de linéarité a un certain moment -> nécessité des simulation numériques
solveur de gravité

\section{l'équation de Poisson}
les différentes méthodes pour la résoudre
Méthode jacobi
méthode multi grille
le pas de temps

\section{Baryon}

système d'équations a résoudres
solveur hydro
partie la plus intensive en calcul
le pas de temps

\section{La chimie}

gestion du refroidissement

\section{radiation}

système d'équations a résoudres
la méthode M1
aton
le cooling
le pas de temps
la mise en place du multi longueur d'onde

\section{gestion du pas de temps}

condition de courant
cosmo
part
freefall
hydro
radiatif

\section{Matériel et parallélisme}

l'évolution du matériel
MPI et courbe de hilbert
CUDA et GPU

\section{Les machines utilisées}

le meso centre de l'UDS
Curie
Titan
Occigen

\section{Gestion des entrées sortie}

le feedback CODA
grosse quantité de données
analyse a distance
conception d'une organisation des données
séparation des champs
structure imposé par la gestion de l'AMR
utilisation de hdf5
écriture parallèle

\section{Potentiel d'optimisation EMMA}

la forme des gathers/scatter
optimisation matérielle -> les prochaines générations de GPU
Opérations coarse sur grille non AMR.
reformatage de l'arbre et découplage de la physique
