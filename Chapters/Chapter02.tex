\chapter{Introduction au modèle numérique}\label{ch:introduction}

Les echelles de temps sont radicalement opposées entre la cosmo qui considère les temps les plus long de l'univers et les progrès informatiques qui vont a une vitesse exponentielle. 
il faut considérer les simulations comme éphémère.

Comment modéliser la reionization?

\section{Les différents types de fluides cosmologique}

Un code de simulation cosmologique a pour vocation principale de suivre l'évolution de différents "fluides", comme la matière noire, la gaz, les étoiles, la radiation ou le champ magnétique.
Ces fluides sont de nature différentes et il n'y a pas de méthode permettant de suivre de manière optimale ces différentes physique.
On distinguera principalement deux catégories de fluides: les fluides collisionnels et les non-collisionnels.

\paragraph{Physique collisionnelle : } elle concerne principalement le gas.

\paragraph{Physique non-collisionnelle : } elle concerne principalement la matière noire ou les étoiles.



\section{Les différents types de codes}

Il existe conceptuellement deux principales façons de suivre un fluide dans l'espace.
Ces deux approches sont dites \emph{Eulérienne} ou \emph{Lagrangienne}.

\paragraph{Representation Lagrangienne : } 
consiste a se placer au point de vue du fluide.
On considère un élément de fluide pouvant se déplacer et/ou se dilater dans l'espace.
On associera généralement les codes utilisant ce type de représentation avec une gestion de la physique sous forme de \emph{particules}.

\paragraph{Representation Eulerrienne : } 
consiste a se placer au point de vue de l'espace.
On considère un élément d'espace et le bilan de matière entrant et sortant de chacune de ses interfaces.
On associera généralement les codes utilisant ce type de représentation avec une gestion de la physique sous forme de \emph{grille}.

EMMA utilise une représentation Lagrangienne pour simuler la matière noire et une représentation Eulerienne pour simuler le gas et la radiation.

En lien direct avec ces deux familles de représentation physique, il existe deux principales famille de codes cosmologique : les codes SPH et les codes AMR.



\paragraph{Smooth Particle Hydrodynamic (SPH) : } représente 

\paragraph{Adaptive Mesh Reffinement (AMR) :  }

%La représentation Lagrangienne la plus populaire (dans le domaine des simulations cosmologiques) est sans doute le \emph{Smouth Particle Hydrodynamics (SPH) }
%Les volumes cosmologiques étant généralement cubique, les éléments de grille le sont généralement aussi.



historique
avantage inconvénient AMR vs SPH
introduction de la grille et de la méthode AMR

\section{Gestion de la grille}

(nécessaire d'être positionné ici car la structure en arbre conditionne plusieur choix par la suite)

Oct tree
gestion du raffinement
cell linked list

\section{Energie noire}
représente la majore partie du contenu de l'univers mais est la plus simple a simuler.
Lien direct avec le facteur d'expansion (mettre ref section).

Il existe deux possibilités pour modéliser l'expansion de l'univers.
La première consiste a considèrer un element de volume $dx$ de taille fixe, et au fur et a mesure que l'univers grandis, a y ajouter des élements.
Le problème et que le cout numérique de la simulation croit entre autre avec le nombre d'éléments que l'on considère.
La seconde possibilité est de faire varier la taille des éléments de calcul avec le facteur d'expansion.
On appellera les longueurs ainsi exprimées des longueurs comobile.

\begin{equation}
r=a r'
\end{equation}

ou $r$ représente une longueur en unités physique et $r'$ en unités comobile.

Ainsi un cube de 10 Mpc physique de coté, pris aujourd'hui, aura une taille de 10 Mpc comobile (cMpc) aujourd'hui, mais aussi a redshift z=9 ou sa taille physique ne sera plus que de de 1Mpc physique.

De plus, il est généralement pratique de normaliser les grandeurs que l'on considère. 

\begin{equation}
r'=\tilde{r}r*
\end{equation}
ou $\tilde{r}$ est la longueur normalisée et $r*$ le facteur de normalisation.


\subsection{Système d'unités supercomobiles}
La généralisation de ce principe a d'autre unités que la longueur est appeler système d'unités supercomobiles.
\citep{martel_convenient_1998}

Longueur:
\begin{equation}
\tilde{r}=\frac{r}{ar_*}
\end{equation}

Densité de matière:
\begin{equation}
\tilde{\rho}=\frac{\rho a^3}{\rho_*}
\end{equation}

Vitesse:
\begin{equation}
\tilde{v}=\frac{av}{v_*}
\end{equation}

Pas de temps:
\begin{equation}
\tilde{dt}=\frac{dt}{a^2t_*}
\end{equation}

Densité d'energie potentielle:
\begin{equation}
\tilde{\Phi}=\frac{a^2 \Phi}{\Phi_*}
\end{equation}

Pression:
\begin{equation}
\tilde{p}=\frac{a^5 p}{p_*}
\end{equation}

Densité d’énergie cinetique:
\begin{equation}
\tilde{\epsilon}=\frac{a^2 \epsilon}{\epsilon_*}
\end{equation}

Densité D’éléments:
\begin{equation}
\tilde{N}=a^3 N r_*^3
\end{equation}

Flux:
\begin{equation}
\tilde{F}=a^4 r_*^2 t_* F
\end{equation}



Facteurs de normalisation:
\begin{equation}
r_*=L (la taille de boite)
\end{equation}

\begin{equation}
\rho_* = \bar{\rho} = \frac{3H_0^2 \Omega_0}{8\pi G}
\end{equation}

\begin{equation}
t_* = \frac{2}{H_0 \sqrt{\Omega_m}}
\end{equation}

\begin{equation}
v_* = \frac{r_*}{t_*}
\end{equation}

\begin{equation}
\Phi_* = \frac{r_*^2}{t_*^2} = v_*^2
\end{equation}

\begin{equation}
p_* = \frac{\rho_* r_*^2}{t_*^2} = \rho_* v_*^2
\end{equation}

\begin{equation}
\epsilon_* = \frac{p_*}{\rho_*} = v_*^2
\end{equation}




\subsection{le pas de temps}

\section{Matière noire}




\subsection{génération des conditions initiales}

méthode
gaussian random noise
théorie des perturbation linéaire
lien avec le spectre de puissance
MUSIC et GRAPHIC
limite la résolution min et max (min en masse et max en espace)


une simulation est limité par sa taille et sa résolution -> ceci définit la plage d'échelle que l'on peut simuler

principes de bases ennoncé dans Pen (1997) and Bertschinger (2001).

    discrétisation de l'espace
    placement des particules sur la grille
    génération d'un bruit blanc
    convolution avec un spectre de puissance connu (celui du CMB)


\subsection{Théorie des perturbation linéaire}

approximation de zeldovich
perte de linéarité a un certain moment -> nécessité des simulation numériques





\subsection{solveur de gravité}
Fluide non collisionnel -> particule\\
On cherche a obtenir le déplacement de ces particules -> 

le système d'équation de Vlasov-Poisson :

\begin{equation}
\begin{cases}

\frac{d{\bf x}_p}{dt} = {\bf v}_p, \\
\frac{d{\bf v}_p}{dt} = -\nabla \phi , \\
\Delta \phi= 4\pi G \rho.

\end{cases}
\label{eq:Ncorps}
\end{equation}


\paragraph{Particule-Particule : } c'est le méthode la plus directe pour calculer l'évolution d'un fluide non collisionnel. 
Elle consiste, pour chaque particule, a sommer les contribution gravitationnelles de toutes les autres particules.
Ce type de code dispose d'une très bonne précision mais la quantité de calcule évolue en ordre $O(N^2)$, ce qui fait que ce type de code est très coûteux.

\begin{equation}
\vec{F}_i=-\sum_{j\neq i} G \frac{G m_i m_j(\vec{r}_i - \vec{r}_j) }{ |\vec{r}_i - \vec{r}_j |^3}
\end{equation}

\paragraph{Particule-tree : } consiste a regrouper les particules loin de la particule courante en un amas aillant une interaction gravitationnelle commune
% (Barnes & Hut 1986) 

\paragraph{Particule-Mesh : } on projette les particules sur une grille
%ENZO code, Bryan & Norman 1998
RAMSES

CIC\\

l'équation de Poisson
\begin{equation}
\Delta \Phi = 4 \pi G \rho
\end{equation}

qui devient en système d'unité supercomobile:
\begin{equation}
\Delta \Phi = 6 a \delta
\end{equation}
avec le contraste de densité: 
\begin{equation}
\delta = \tilde{\rho} / < \tilde{\rho} > - 1 
\end{equation}



les différentes méthodes pour la résoudre\\
FFT\

Méthode jacobi\\
\begin{equation}
\Delta \Phi -S = \frac{d \Phi}{dt}
\end{equation}

Over relaxation\\
Gauss siedel
méthode multi grille\\
le pas de temps\\

\section{Baryon}

système d'équations a résoudres
solveur hydro
partie la plus intensive en calcul
le pas de temps

\section{La chimie}

gestion du refroidissement

\section{radiation}

système d'équations a résoudre
la méthode M1
aton
le cooling
le pas de temps
la mise en place du multi longueur d'onde

\section{gestion du pas de temps}

condition de courant
cosmo
part
freefall
hydro
radiatif

\section{Matériel et parallélisme}

l'évolution du matériel
MPI et courbe de hilbert
CUDA et GPU

\subsection{Les machines utilisées}

le meso centre de l'UDS
Curie
Titan
Occigen

\subsection{Gestion des entrées sortie}

le feedback CODA
grosse quantité de données
analyse a distance
conception d'une organisation des données
séparation des champs
structure imposé par la gestion de l'AMR
utilisation de hdf5
écriture parallèle

\subsection{Potentiel d'optimisation EMMA}

la forme des gathers/scatter
optimisation matérielle -> les prochaines générations de GPU
Opérations coarse sur grille non AMR.
reformatage de l'arbre et découplage de la physique
