\chapter{Les Halos}

L'étude des simulation se fait souvent halos par halos.
Quand on cherche a déterminer es propriétés des galaxies, un paramètre important est leur masse.
Comme le gaz suit la dynamique des baryons, les galaxies sont situées dans les surdensités de matière noire.


\section{La detection des halos}

L'objectif d'un halo finder est de detecter les surdensités dans le champs de matière noire.
Comme nous l'avons vu plus tot, (TODO ref) le champ de matière noire utilise une représentation sous forme de particules.

J'ai principalement utilisé l'algoritme Friend Of Friend pour detecter les sur densités.

FOF retourne une liste de position de halo, une liste permettant de lier les particules detecté aux halo.

\section{Le rayon de Viriel}
Approximation par le R200:
\begin{equation}
R_{200}=\frac{3\cdot M_{FoF} }{4\pi\cdot 200 \bar{\rho} }
\end{equation}


\section{Le problème de la forme des halos}
Fortement non viriallisé a z=6\\
beaucoup de dynamique et merger dans les filemments

\section{association dans le R200}

On cherchera a associer les galaxies  (surdensité d'étoiles) aux halos.

La première méthode consiste a utiliser l'approximation du R200.
Le méthode est la suivante:

\begin{itemize}
\item générer un KDtree sur les étoiles
\item Faire une recherche sphérique autour de la positions des halo, sur un rayon de R200
\end{itemize}

Il sera possible d'utiliser une méthode comparable sur la grille, en utilisant les centres de cellule a la place de la position des étoiles.

Comme les particules de matière noire  données par FoF ne represente pas le meme volume, dans un soucis de cohérence, on appliquera cette méthodes a la matière noire également.


\section{association "fine"}
Dans le but d'améliorer ces problèmes d'identification dans les halos fortement non virialisé, j'ai développer une méthode pour associer plus précisement les halos et la grille.


la méthode consiste a :
\begin{itemize}
\item generer un KDtree sur la grille
\item pour chaque particule d'une halo, trouver la plus proche cellule
\item filtrer les cellules doublons
\end{itemize}