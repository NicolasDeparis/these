\chapter{La composante stellaire}

deffinition d'une étoile\\


\section{Les differentes phases de la vie d'une étoile}


le diagram HR



\subsection{naissance}
effondrement hierarchique\\
lien avec la densité\\

\subsection{sequance principale}
majore partie du temps\\

equilibre hydrostatique entre gravitation et reaction de fusion nucléaire nucléaire\\

equation simplifié du processus de fusion:
\begin{equation}
4p \leftrightarrow He^4 + 2e+ + 2\nu + E
\end{equation}

%develloper le cycle proton proton ?



materiaux de base est l'hydrogene\\
plasma donc hydrogène ionisé 
formation d'élement lourd (<Fe)\\
enrichissement du milieu


\subsection{mort}
Consomation du materiaux de base

géante rouge\\
naine blanche\\
trou noir\\
supernovae\\
formation d'elements lourd (>Fe)



lien entre les différents solveurs en fonction du stade évolutif

En fonction des echelles de travail, nous considererons soit les etoiles individuelles soit une population stellaire.

\section{La formation stellaire}

\subsection{La loi de schmidt-kennicut}
Loi observationnelle \\
conversion densité surfacique vers densité 3D\\
rho 1.5\\
temps de free fall\\

Seuil en densité \\
\begin{equation}
	flag = 
  \begin{cases}
      True, & \text{if } \rho > \rho_{thresh}\\
      False,              & \text{otherwise}
  \end{cases}
\end{equation} 




\begin{equation}
	\rho_{thresh} = max\left(  \delta_{in} \bar{\rho}, \rho_{in} a^3 \right)
\end{equation} 

ou $\delta_{in}$ et $\rho_{in}$  sont respectivement les parametre d'overdensité et de densité physique.
$\delta_{in}$est exprimé en unité comobile et est donc constant dans le temps en unité du code
 $\rho_{in}$ est exprimé en unité physique (en atome par metre cube), sa valeur evolue dans le temps du point de vue des unité du codes.
 
Dans les cellules au dessus du seuil on considère une SFR de:

\begin{equation}
	SFR = \epsilon \frac{\rho_g}{t_{ff}}
    \label{eq_sfr}
\end{equation}


avec  $\epsilon$ le parametre d'efficacité de formation stellaire , $\rho_g$ la densité de gas locale, et le temps de chute libre:
\begin{equation}
t_{ff} = \sqrt{\frac{3\pi}{32G\rho_g}}
\end{equation}

\begin{equation}
	M_{star} = SFR . dv .dt 
\end{equation}

determination de la valeur de 55\\

tirage aléatoire\\
loi de Poisson\\

resolution en masse\\



\section{La vie radiative}

injection d'énergie dans le solveur radiatif, ok mais combien?\\
calibration energetique et Starburst99\\


$
    S = 
\begin{cases}
    S_0 ,         & \text{if } t < t_{life}\\
    S_0.t^{-4},   & \text{if } t_{life} \leq t < 100.t_{life} \\
    0,   & \text{if } 100t_{life} \leq t
\end{cases}
$


\begin{table}
\begin{tabular}{|l|l|}
  \hline
	$<h\nu>$	&  23.42 eV \\
	$\alpha_e$	&  $2.35.10^{-22}$ m$^2$ \\
	$\alpha_i$	&  $1.82.10^{-22}$ m$^2$ \\
  \hline
\end{tabular}
\caption{Photon properties
\label{tab_photon}}
\end{table}

Tab \ref{tab_photon} caracteristique des photons.




We found a mean energy of $<h\nu> = 23.42$ eV,
an energy weighted cross section of
$\alpha_e = 2.35.10^{-22}$ m$^2$
and a number weighted cross section of
$\alpha_i = 1.82.10^{-22}$ m$^2$



Masse de la population\\
produit en croix pour correspondre a la masse dans simu\\
integration seulement sur energy ionisante\\

Multigroupe frequence\\
multigroup temporel\\


\section{le problème de la masse des étoiles}

le paramètre de masse des étoiles change la reionization\\
effet numérique\\
le rayonnement est piégé dans les cellules\\


\section{Les supernovae}

\subsection{le modele theorique}
Les etoiles de plus de 8mo ewploses en SN en injectant 1e51 erg dans le milieu\\
Cette injection limite fortement la formation stellaire dans le milieu.\\
modele sous grille\\



Les differentes phases
\begin{itemize}
\item expansion adiabatique
\item snowplow
\end{itemize}

\subsection{ differentes implementations existantes}


\subsection{Test numérique (Sedov)}

Le test de Sedov cherche a reproduire une explosion parfaite.
Il consiste a relacher instantanemant une quantité dénergie $E$ dans un milieu homogène de densité $\rho$ et de température $T$.

Sedov a demonter en 1959 que :
\begin{equation}
r_{(t)}=\left( \frac{E_0}{\alpha \rho_0 }\right)^{1/5} t^{2/5}
\end{equation}



Ce brusque changement dans l'etat du systeme créer une discontinuité que le solveur va devoir gérer.



OK\\
mais pas en cosmo




\subsubsection{Sedov evolution}

injection thermique simple\\
test en 256**3 sans raffinement\\

\begin{figure}[bth]
        \includegraphics[width=.95\linewidth]{img/03/sedov/sedov_evol_8_den_lin.pdf} 
		\includegraphics[width=.95\linewidth]{img/03/sedov/sedov_evol_8_pres.pdf} 
		\includegraphics[width=.95\linewidth]{img/03/sedov/sedov_evol_8_vel.pdf} 
        \caption{Test de Sedov, evolution des differentes variables d'etats}
 		\label{fig:}
\end{figure}


\subsubsection{Sedov comparaison}

test en 128**3 avec raffinement, 3 niveaux

mise en place du raffinement :
raffinement sur le gradient 


\begin{figure}[bth]
        \includegraphics[width=.95\linewidth]{img/03/sedov/slice_therm1.pdf} 
		\includegraphics[width=.95\linewidth]{img/03/sedov/slice_therm4.pdf} 
		\includegraphics[width=.95\linewidth]{img/03/sedov/slice_kin.pdf} 
        \caption{Test de Sedov}
 		\label{fig:}
\end{figure}

\begin{figure}[bth]
        \includegraphics[width=.95\linewidth]{img/03/sedov/slice_th_1raf.pdf} 
        \caption{Test de Sedov, raffinement (mettre la color map) }
 		\label{fig:}
\end{figure}

\begin{figure}[bth]
        \includegraphics[width=.95\linewidth]{img/03/sedov/sedov_comp_profile_den.pdf} 
		\includegraphics[width=.95\linewidth]{img/03/sedov/sedov_comp_profile_pres.pdf} 
		\includegraphics[width=.95\linewidth]{img/03/sedov/sedov_comp_profile_vel.pdf} 
        \caption{Test de Sedov, evolution des differentes variables d'etats}
 		\label{fig:}
\end{figure}




\subsection{Mes Implémentations}

\begin{equation}
e_{SN} = E_{SN}/8
\end{equation}

Then this energy is used to change the gas velocity by using:
\begin{equation}
    \Delta \overrightarrow{v_{gas}} = \sqrt{\frac{2e_{SN}}{\rho_g.dV}} \overrightarrow{u}
    \label{eq_sn_direct}
\end{equation}

le pas de temps\\
\section{test}
fonction de luminosité 




\section{Influence du feedback stellaire sur la reionization}




\subsection{fraction d'echapement}à

