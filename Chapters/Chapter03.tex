\chapter{La composante stellaire}

définition d'une étoile\\

Maintenant que nous avons survolé toutes les physiques a l’œuvre dans ce type de simulation, concentrons nous sur les sources lumineuse.
comme nous avons vu, il existe deux types de sources, les étoiles et les quasars. %TODO ref
Je n'ai considérer que la partie stellaire.



\section{Les differentes phases de la vie d'une étoile}

le diagram HR


La nucléosynthèse primordiale a créer peu de métaux. 
A haut redshift, les metaux etaient peu disponible.
les métaux permettent une meilleur emmission radiative, et donc un meilleur refroidissent.
si le refroidissment est meilleur, l'equilibre hydrostatique penche an faveur de plus grosse étoiles.
POPIII, IMF top Heavy, étoiles  primordiales

plus un étoiles est grosse, plus son spectre sera énergétique et plus la portion de spectre ionisant sera important.

A la fin de se vie, une étoile a consommée la plus grande partie de son hydrogène disponible.
l'équilibre radiatif est rompu et l'étoile s'éffondre.
cette augmentation de la pression amorce une serie de fusion nucléaire d’élément plus lourd que l’hydrogène.
cycle CNO
Une fois arrivé au Fe, il devient coûteux de continuer a fusionner.
FE le plus stable



Géante rouge

En fonction de la masse:
naine blanche (pression de dégénérescence des électrons)
etoile a neutron (pression de dégénérescence des neutrons)
trou noir (singularité)



Différents type de supernovae

I -> binaire accrétante du compagnon -> passage au dessu de la limite.
II -> l'étoile est assez massive au départ (M>8Mo)


Les supernovae ont été introduite dans les simulations cosmologiques pour contre ballancer "l'overcooling probleme".
Sans l'introduction d'énergie dans le gaz par les supernovae, le gas s'éffondre de manière importante et créer un nombre élevé d'étoiles.
cela mène a un taux de formation stellaire trop important par rapport a ce qui est observé.



les superbubbles
A la manière de la percolllation des bulles de HII, les bulles de supernovae 








Les étoiles se trouvent aux centre de la simulation.
Le probleme du modèle sous grille
// concept de particule puits 
En effet, créer une étoiles consiste a transformer une partie du gaz en particule.
Cette particule sera ensuite gérée par le solveur Ncorps, et servira de source au solveur radiatif.
a la fin de sa vie, l'étoile va injecter de l'énergie dans le solveur hydro.


Seul la partie du spectre capable de ioniser l'hydrogène est considérée. E>13.6eV

\subsection{naissance}
effondrement hierarchique\\

masse de Jeans
lien entre vitesse du son et vitesse de chute libre
si la vitesse de chute libre est supérieur a la vitesse du son, le milieu n'a pas le temps de répondre a son effondrement, la gaz s'effondre et forme une étoile.

\begin{equation}
t_{sound} = \frac{R}{C_s}
\end{equation}

\begin{equation}
t_{ff} = \frac{1}{\sqrt{G \rho}}
\end{equation}

\begin{equation}
t_{ff} = t_{sound}
\end{equation}




lien avec la densité\\

Formation dans l'H moléculaire mais pas dans les simu

\subsection{Pop III}


a haut redshift:

TRes peux de métaux
tres peu de ligne de refroisdissment
étoiles plus grosse
 temps de vie court




\subsection{sequence principale}
majore partie du temps\\

equilibre hydrostatique entre gravitation et reaction de fusion nucléaire nucléaire\\

equation simplifié du processus de fusion:
\begin{equation}
4p \leftrightarrow He^4 + 2e+ + 2\nu + E
\end{equation}

%develloper le cycle proton proton ?



materiaux de base est l'hydrogene\\
plasma donc hydrogène ionisé 


\subsection{mort}
Consomation du materiaux de base

géante rouge\\
naine blanche\\
trou noir\\
supernovae\\
formation d'elements lourd (>Fe)
enrichissement du milieu



lien entre les différents solveurs en fonction du stade évolutif

En fonction des echelles de travail, nous considererons soit les etoiles individuelles soit une population stellaire.

\section{les modèles de population stellaire}

\subsection{Fonction de masse Initiale}
notion de population stellaire

Salpeter
Krupa

Fonction de masse IMF Top Heavy

durant les calibrations: difficulté a reioniser avec Salpeter -> passage a top heavy -> justification 

\subsection{starburst99}
paramètre d'entrée
sorties



\section{La formation stellaire}

\subsection{La loi de schmidt-kennicut}
Loi observationnelle \\
conversion densité surfacique vers densité 3D\\
rho 1.5\\
temps de free fall\\

Seuil en densité \\
\begin{equation}
	flag = 
  \begin{cases}
      True, & \text{if } \rho > \rho_{thresh}\\
      False,              & \text{otherwise}
  \end{cases}
\end{equation} 




\begin{equation}
	\rho_{thresh} = max\left(  \delta_{in} \bar{\rho}, \rho_{in} a^3 \right)
\end{equation} 

ou $\delta_{in}$ et $\rho_{in}$  sont respectivement les paramètre de surdensité et de densité physique.
$\delta_{in}$est exprimé en unité comobile et est donc constant dans le temps en unité du code
 $\rho_{in}$ est exprimé en unité physique (en atome par metre cube), sa valeur evolue dans le temps du point de vue des unité du codes.

determination de la valeur de 55\\ 
 
 
Dans les cellules au dessus du seuil on considère une SFR de:

\begin{equation}
	SFR = \epsilon \frac{\rho_g}{t_{ff}}
    \label{eq_sfr}
\end{equation}


avec  $\epsilon$ le paramètre d'efficacité de formation stellaire , $\rho_g$ la densité de gaz locale, et le temps de chute libre:
\begin{equation}
t_{ff} = \sqrt{\frac{3\pi}{32G\rho_g}}
\end{equation}

\begin{equation}
	M_{star} = SFR . dv .dt 
\end{equation}

resolution en masse\\

Nous avons donc a ce stade la masse totale de gaz a convertir en étoile.
Il peux etre tres couteux de générer pour chaque cellule éligible, et a chaque pas de temps, une nouvelles particule stellaire (le nombre de particule peux rapidement exploser.
Nous adoptons une approche probabiliste.
Nous définissons une masse d'étoiles $m_{star}$ qui correspondra a notre "quanta stellaire".
toutes les étoiles aurons donc la même masse.
Et nous tirons le nombre de quanta a ajouter aléatoirement dans une lois de Poisson.

\begin{equation}
	P(N) = \frac{\lambda^N}{N!} e^{-\lambda}
\end{equation}

Ou $\lambda$ correspond au nombre de particule moyen a créer dans la cellule ($\lambda \frac{ M_{star}}{m_{star}}$
Étant donné le grand nombre de tirage cette loi est en moyenne valide.

La masse des étoiles est calculée d'une manière comparable a la masse d'une particule de matière noire.
la masse d'une étoile correspond a la masse moyenne de gaz dans une cellules d'un certain niveau pouvant aller du niveau coarse $m_star = M_{DM} \frac{\Omega_b}{\Omega_m}$ au niveau plusieurs niveau raffiné.


LEs étoiles créées auront une vitesse aléatoire pour éviter les effets de "collier".




\section{La vie radiative}

injection d'énergie dans le solveur radiatif, ok mais combien?\\
calibration energetique et Starburst99\\


$
    S = 
\begin{cases}
    S_0 ,         & \text{if } t < t_{life}\\
    S_0.t^{-4},   & \text{if } t_{life} \leq t < 100.t_{life} \\
    0,   & \text{if } 100t_{life} \leq t
\end{cases}
$


\begin{table}
\begin{tabular}{|l|l|}
  \hline
	$<h\nu>$	&  23.42 eV \\
	$\alpha_e$	&  $2.35.10^{-22}$ m$^2$ \\
	$\alpha_i$	&  $1.82.10^{-22}$ m$^2$ \\
  \hline
\end{tabular}
\caption{Photon properties
\label{tab_photon}}
\end{table}

Tab \ref{tab_photon} caracteristique des photons.


intégration de l'énergie est de la section efficace


We found a mean energy of $<h\nu> = 23.42$ eV,
an energy weighted cross section of
$\alpha_e = 2.35.10^{-22}$ m$^2$
and a number weighted cross section of
$\alpha_i = 1.82.10^{-22}$ m$^2$



Masse de la population\\
produit en croix pour correspondre a la masse dans simu\\
integration seulement sur energy ionisante\\

Multigroupe frequence\\
multigroup temporel\\


\section{le problème de la masse des étoiles}

le paramètre de masse des étoiles change la reionization\\
effet numérique\\
le rayonnement est piégé dans les cellules\\


\section{Les supernovae}

\subsection{le modele theorique}
Les etoiles de plus de 8mo ewploses en SN en injectant 1e51 erg dans le milieu\\
Cette injection limite fortement la formation stellaire dans le milieu.\\
modele sous grille\\



Les differentes phases
\begin{itemize}
\item expansion adiabatique
\item snowplow
\end{itemize}

\subsection{ differentes implementations existantes}


\subsection{Test numérique (Sedov)}

Le test de Sedov cherche a reproduire une explosion parfaite.
Il consiste a relacher instantanemant une quantité dénergie $E$ dans un milieu homogène de densité $\rho$ et de température $T$.

Sedov a demonter en 1959 que :
\begin{equation}
r_{(t)}=\left( \frac{E_0}{\alpha \rho_0 }\right)^{1/5} t^{2/5}
\end{equation}



Ce brusque changement dans l'etat du systeme créer une discontinuité que le solveur va devoir gérer.



OK\\
mais pas en cosmo




\subsubsection{Sedov evolution}

injection thermique simple\\
test en 256**3 sans raffinement\\

\begin{figure}[bth]
        \includegraphics[width=.95\linewidth]{img/03/sedov/sedov_evol_8_den_lin.pdf} 
		\includegraphics[width=.95\linewidth]{img/03/sedov/sedov_evol_8_pres.pdf} 
		\includegraphics[width=.95\linewidth]{img/03/sedov/sedov_evol_8_vel.pdf} 
        \caption{Test de Sedov, evolution des differentes variables d'etats}
 		\label{fig:}
\end{figure}


\subsubsection{Sedov comparaison}

test en 128**3 avec raffinement, 3 niveaux

mise en place du raffinement :
raffinement sur le gradient 


\begin{figure}[bth]
        \includegraphics[width=.95\linewidth]{img/03/sedov/slice_therm1.pdf} 
		\includegraphics[width=.95\linewidth]{img/03/sedov/slice_therm4.pdf} 
		\includegraphics[width=.95\linewidth]{img/03/sedov/slice_kin.pdf} 
        \caption{Test de Sedov}
 		\label{fig:}
\end{figure}

\begin{figure}[bth]
        \includegraphics[width=.95\linewidth]{img/03/sedov/slice_th_1raf.pdf} 
        \caption{Test de Sedov, raffinement (mettre la color map) }
 		\label{fig:}
\end{figure}

\begin{figure}[bth]
        \includegraphics[width=.95\linewidth]{img/03/sedov/sedov_comp_profile_den.pdf} 
		\includegraphics[width=.95\linewidth]{img/03/sedov/sedov_comp_profile_pres.pdf} 
		\includegraphics[width=.95\linewidth]{img/03/sedov/sedov_comp_profile_vel.pdf} 
        \caption{Test de Sedov, evolution des differentes variables d'etats}
 		\label{fig:}
\end{figure}




\subsection{Mes Implémentations}

\begin{equation}
e_{SN} = E_{SN}/8
\end{equation}

Then this energy is used to change the gas velocity by using:
\begin{equation}
    \Delta \overrightarrow{v_{gas}} = \sqrt{\frac{2e_{SN}}{\rho_g.dV}} \overrightarrow{u}
    \label{eq_sn_direct}
\end{equation}

le pas de temps\\
\section{test}
fonction de luminosité 




