\section{Intro}

Cette lettre s'inscrit dans une strategie de travail a long terme.
Elle a pour vocation de présenter la simulation "CODA II EMMA" ainsi que les premiers resultats concernant le lien entre la période de réionisation et l'epoque actuelle

Cette étude montre que les halos les plus massifs a z=0 sont réionisé plus tôt que le reste de l'Univers,  

Lettre donc partie courte.


Objectif connaitre le Z reio des halo en fonction de leur masse.
Sur grosse simu donc beaucoup de stats


\subsection{Présentation de la simulation CODA II EMMA}

\subsubsection{Conditions initiales}

Les conditions initiales ont été générée par la collaboration CLUES (Constrained Local UniversE Simulations).
L'objectif est de retrouver dans la simulation des structures aillant des caractéristiques proches de ce qui est observé dans l'univers local.
On cherchera par exemple a obtenir un couple Andromède - Voie Lacté avec des masses, distances et vitesses relative en accord avec les contraintes actuelles.


Le volume de $\left( 64/h cMpc \right) ^3$ est échantillonné par $2048^3$ particules de matière noire.
Ce qui mêne a une résolution en masse $3.4 \cdot 10^6 M_\odot$ et une résolution spatiale de 46 ckpc sur la grille coarse.
Ces paramètres permettent d'explorer la gamme de masse de halos compris entre $10^8 M_\odot$ et  $10^{13}M_\odot$.



différences avec 
CODA I RAMSES CUDATON 

Même IC 

8x moins résolue en masse 
Mais mieux résolue en dx





objectif comparaison entre les 2 


\subsubsection{Setup}

1 Simu Pure DM avec Gadget 

1 simu full coupled RHD avec EMMA

32768 CPU
4096 GPU
TITAN


\subsection{projection a z=0}




les 2 méthodes :

- Centre de masse pris dans le merger tree
- moyenne du t des particules



\subsection{Résultats}


