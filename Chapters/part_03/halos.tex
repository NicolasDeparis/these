\chapter{Les galaxies}
\label{sec:galaxies}

Les étoiles se forment rarement de manière isolée mais généralement par paquet dans des régions bien définies.% que l'on nomme galaxies.
Ces zones ont une densité de gaz particulièrement élevée, et disposent donc de suffisamment de matériaux pour amorcer la formation stellaire.
Comme la densité de gaz suit la distribution de matière noire, c'est grâce à cette dernière que l'on détecte les galaxies.
L'idée est d'utiliser un "détecteur de halo" (ou "halo finder") pour détecter les surdensités de matière noire, auquel on cherchera ensuite à associer la composante baryonique.

Cette partie a pour objectif de présenter les méthodes de détection des galaxies, pour pouvoir ensuite mener une série d'études statistiques.
Nous analyserons différentes propriétés des galaxies en fonction de la masse de leur halo hôte.
Nous nous intéresserons à des propriétés telle que la fraction de masse baryonique, le taux de formation stellaire, le taux d'éjection de matière en fonction des supernovæ, ou encore la fraction d'échappement du rayonnement.

%L'étude des simulation se fait souvent halos par halos.
%Quand on cherche a déterminer es propriétés des galaxies, un paramètre important est leur masse.
%Comme le gaz suit la dynamique des baryons, les galaxies sont situées dans les surdensités de matière noire.

\section{La détection des halos}
La première étape consiste à détecter les halos.
Comme nous l'avons vu plus tôt (voir sec \ref{sec:solverDM}), le champ de matière noire utilise une représentation sous forme de particules.
L'objectif d'un halo finder est de détecter les surdensités dans ce champs de particule.

Il existe différents techniques, mais j'ai principalement utilisé l’algorithme \ac{FOF} pour détecter les sur-densités et plus particulièrement \textit{pFOF} une implémentation parallèle de \ac{FOF} \citep{2014A&A...564A..13R}.
\textit{pFOF} lie les particules regroupées en surdensité, en utilisant une taille de lien caractéristique.
Il retourne ensuite une liste de groupes de particules et une liste de positions déterminées par le centre de masse des particules de chaque halos.
%FOF retourne une liste de position de halo, une liste permettant de lier les particules détecté aux halos.
%Il existe plusieurs possibilités pour ensuite déterminer l'étendue spatiale des halos.

\subsection{Le rayon de Viriel}
La possibilité la plus directe pour déterminer l'étendue spatiale d'un halo consiste à utiliser l'approximation du $R_{200}$ \citep{1997ApJ...490..493N}.
Elle consiste à considérer autour du centre de masse, une sphère aillant une densité moyenne de $200$ fois la densité moyenne de l'univers $\bar{\rho}$.
Le $R_{200}$ est défini comme : 
\begin{equation}
R_{200}= \left( \frac{3\cdot M_{FoF} }{4\pi\cdot 200 \bar{\rho}} \right)^{1/3}
\end{equation}


\subsection{Association dans le R$_{200}$}

%Il faut ensuite y associer les autres champs physique contenus par la grille 
%La première méthode consiste a utiliser l'approximation du R200.
%Il faudra associer, la matière noire, le gaz et les étoiles compris a l'intérieur d'une sphere de rayon R200.

%Il faut ensuite y associer les autres champs physique contenus par la grille 

%Le méthode est la suivante:
%
%\begin{itemize}
%\item générer un KDtree sur les particule de matière noire / les étoiles / la grille AMR
%\item Faire une recherche sphérique autour de la positions des halo, sur un rayon de R200
%\item stocker les indices dans une structure
%\end{itemize}

Nous avons donc à ce stade une position et une étendue pour chaque halo.
On cherchera ensuite à y associer les baryons formant les galaxies.
Pour se faire j'ai utilisé un KD-tree, qui est un arbre permettant d'effectuer des recherches spatiales de manière optimisées.
A l'aide d'un arbre généré sur les étoiles, j'ai associé a chaque halo toutes les étoiles présentes a l'intérieur de son $R_{200}$.
Nous chercherons également a obtenir des informations sur les champs physique contenus pas la grille AMR.
Il sera possible d'utiliser un KD-tree pour déterminer quelles sont les cellules qui se trouvent dans le $R_{200}$ du halo.
Il suffit pour cela de générer un arbre sur les centres des cellules à la place de la position des étoiles.
%Comme les particules de matière noire  données par FoF ne represente pas le meme volume, dans un soucis de cohérence, on appliquera cette méthodes a la matière noire également.
A ce stade nous avons donc la possibilité, pour chaque halo, de connaître n'importe qu'elle grandeurs comprissent dans son $R_{200}$.

\subsection{Le problème de la forme des halos}

%Fortement non viriallisé a z=6\\
%beaucoup de dynamique et merger dans les filemments

L'approximation du $R_{200}$ est valide et a fait ses preuves dans le cas de halos virialisés.  %TODO ref
Cependant, à haut redshift (z>5) les halos sont encore en formation et peuvent être fortement non virialisé.
Il devient alors difficile pour \ac{FOF} de séparer les différentes sous parties d'une même sur-densité.
La figure \ref{fig:part_halo} illustre particulièrement bien ce problème de détection.
La sur-densité détectée par \ac{FOF} est loin d'être sphérique, et l'approximation du $R_{200}$ n'est pas valide dans ce cas.
Du fait de l'effondrement hiérarchique de la matière noire, changer la longueur de lissage ne changera pas le problème et ne ferait que le reporter à d'autres échelles.

\begin{figure}
	\centering
	\includegraphics[width=.75\textwidth]{img/03/part_halo_R200.pdf} 
    \caption[Détection des halos]{La détection des halos par \ac{FOF} peux être difficile au moment de la réionisation car les structure peuvent ne pas être virialisée.
    Les points bleus représentent les particules de matière noire et le cercle le $R_{200}$.
 	\label{fig:part_halo}}
\end{figure}


\subsection{Association "fine"}

Dans le but d'améliorer ces problèmes d'identification dans les halos fortement asphérique, j'ai développé une méthode pour associer plus précisément la grille aux halos.
Cette méthode consiste à rechercher pour chaque particule de matière noire, sa plus proche cellule.
Malgré la recherche effectué à l'aide d'un KD-tree, cette méthode est évidemment bien plus longue qu'une simple recherche dans la sphere.
La figure \ref{fig:R200_fine} présente la détection des cellules entre les deux méthodes pour le halo de la figure \ref{fig:part_halo}.

%\begin{itemize}
%\item générer un KDtree sur la grille
%\item pour chaque particule d'une halo, trouver la plus proche cellule
%\item filtrer les cellules doublons
%\end{itemize}



\begin{figure}
	\subfloat[$R_{200}$]{\includegraphics[width=.45\linewidth]{img/03/part_cells.pdf} }
	\subfloat[\textit{fine}]{\includegraphics[width=.45\linewidth]{img/03/part_cells_fine.pdf} }
    \caption[Méthodes d'association matière noire - grille]{Cellules associées au halo présenté sur la figure \ref{fig:part_halo} dans le cas de l'estimation du $R_{200}$ à gauche et dans le cas de la méthode fine à droite.       
 	\label{fig:R200_fine}}
\end{figure}


\subsection{Association des petits halos}
Pour les petits halos, peu résolus, la taille des cellules peu être grandes par rapport au $R_{200}$.
Pour minimiser l'erreur commise, il est nécessaire d’estimer l'intersection entre le cube de la cellule et la sphère du halo.
J'ai résolu ce problème géométrique en projetant la grille \ac{AMR} sur une grille régulière de résolution arbitraire.
Cette méthode permettra d'associer des sous parties de cellules aux halos en approximant l'intersection avec la cellule et améliorera la precision globale de l'association (cf figure \ref{fig:intersec}).

\begin{figure}
	\centering
	%TODO refaire cette image
	\includegraphics[width=.45\linewidth]{img/03/intersec.png}
    \caption[Projection des petits halo]{Les plus petits halos peuvent avoir des tailles comparable aux cellules.
    Il est nécessaire d’estimer l'intersection cube/sphère.
 	\label{fig:intersec}}
\end{figure}

\clearpage
\section{Étude de la composition des halos en fonction du feedback de supernovae}

A ce stade, nous avons définis la position et la taille de tout les halos et définis les galaxies par associations.
%Maintenant que nous pouvons associer toutes la physique au sein de chaque halo, 
Nous pouvons étudier l'influence du feedback de supernovae sur différentes caractéristiques des galaxies.
%différentes caractéristiques en fonction du type de feedback.

Pour ce faire réalisé quatre simulation de $\left(8/h cMpc \right)^3$ avec des caractéristiques identiques à celles présentées en section \ref{sec:pres_simu}.
Parmi ces simulations, il y en a :
\begin{itemize}
\item une sans rayonnement ni supernovae (NoFEED),
\item une sans supernovae (NoSN),
\item une avec supernovae - modèle thermique (Thermal),
\item une avec supernovae - modèle cinetique (Kinetic).
\end{itemize}


\subsection{Approche visuelle}
\label{sec:snmaps}

Réaliser des cartes de champs permet d'aborder la problème.
La figure \ref{fig:halo} présente le halo le plus massif ($M\approx10^{11} M_\odot$) extrait des trois simulations avec rayonnement et un modèle de supernovæ différent (cf section \ref{sec:SNmodel}).
On y observe que le modèle cinétique est capable de générer bulles de gaz chaud significativement plus grande que le modèle thermique.
De plus on observe dans la densité des coquilles dues aux vents généré, autour du halo central dans le cas du modèle cinétique.

\begin{figure}
		\includegraphics[width=.95\linewidth]{img/03/halos.pdf}
        \caption[Influence du modèle de supernovae sur la forme des halos]{Influence du modèle de supernovae sur la distribution de densité et de température d'un halo.
        le cercle représente le $R_{200}$.
        Le modèle cinétique permet la création de vents formant des coquilles autour du halo.
        De plus la région chaude est bien plus large que dans le cas du modèle thermique.
 		\label{fig:halo}}
\end{figure}


\subsection{Les fonctions de masse}
\label{sec:hmf}

Sur la figure \ref{fig:ghmf} sont présentés les fonction de masses des halos et des galaxies, en cumulative (n>M).
La fonction de masse des halos n'est pas impactée par le feedback de supernovae dans nos modèles.
Par contre, la fonction de masse des galaxies est significativement réduite avec le feedback.

\begin{figure}
		\includegraphics[width=.95\linewidth]{img/03/ghmf.pdf}
        \caption[Fonctions de masses G/HMF]{Fonction de masse des galaxies (\ac{GMF}) et des halos (\ac{HMF}) pour différents type de feedback de supernovae.
        Le feedback n'a pas d'influence directe sur la function de masse des halos, mais a un impact mesurable sur la formation stellaire.
 		\label{fig:ghmf}}
\end{figure}

%\subsection{La masse d'étoiles en fonction de la masse du halo}
%Plus un halo est massif, plus celui ci va contenir d'étoiles.


\subsection{La formation stellaire en fonction de la masse du halo}
\label{sec:sfr_halo}
Connaissant, à un instant donné, les étoiles appartenant à chaque halo, il est possible de déterminer leurs ages, et donc de connaître l'histoire de formation stellaire de chaque halo.
%Nous allons voir que la masse 
%Il existe deux facons de voir 
Ce problème peux être abordé de deux façons.
Dans un cas l'objectif est de déterminer la \ac{SFR} moyenne d'un halo d'une certaine masse.
Dans l'autre, on cherche à mesurer la contribution d'une classe de masse a la \ac{SFR} globale.
Le lien entre les deux, n'est autre que la \ac{HMF} présentée à la section \ref{sec:hmf}.
Les résultats obtenus sont présentés sur la figure \ref{fig:sfr_halo}.
On observe que pour le feedback cinétique, la \ac{SFR} moyenne est réduite pour les halos les plus lourd, ce qui n'est pas le cas pour le feedback thermique où la \ac{SFR} a tendance à être réduite pour les petites masses de halos.
Sur les courbes de \ac{SFR} totale, on observe que les halos de masses $M\approx10^{10}M_\odot$ contribues le plus a la formation stellaire dans ces simulations.

\begin{figure}
		\includegraphics[width=.95\linewidth]{img/03/SFR_halo_avg.pdf}
		\includegraphics[width=.95\linewidth]{img/03/SFR_halo_tot.pdf}
        \caption[SFR en fonction de la masse du halo]{SFR en fonction de la masse du halo moyenne en haut et total en bas.
        }
 		\label{fig:sfr_halo}
\end{figure}



\subsection{La fraction baryonique}
\label{sec:baryon_frac}
La fraction baryonique est une grandeur importante pour caractériser un halo.
En effet, le rayonnement n'interagit qu'avec les baryons et plus un halo aura de baryon, plus le rayonnement aura de difficulté a s'en échapper.

\begin{equation}
f_b = \frac{M_* + M_{gas} }{M_{DM} + M_* + M_{gas} }
\end{equation}

\begin{figure}
		\includegraphics[width=.95\linewidth]{img/03/baryon_frac.pdf}
        \caption[Fraction baryonique]{Fraction baryonique en fonction de la masse du halo et des feedback.
 		\label{fig:bfrac}}
\end{figure}

La figure \ref{fig:bfrac} présente les résultats obtenus.
On y observe une décroissance de la fraction baryonique pour les halos les plus massifsavec le feedback cinétique.
Cette décroissance est similaire à celle observée sur la figure \ref{fig:sfr_halo}.
La \ac{SFR} est réduite car les baryons sont expulsés par le feedback.


\clearpage
\section{Étude à la surface des halos}

Nous venons d'observer que le feedback de supernovae a un impact sur la composition interne des halos et nous avons interprété ce changement par une expulsion du gaz en dehors des halos.
Nous cherchons ici a quantifier les flux entrants et sortant des halos en fonction de leurs masses.
Je présenterai la méthodologie utilisé pour la détection des flux, ainsi que les résultats obtenus pour les flux hydrodynamiques.
Dans le but de comprendre pourquoi l'histoire d'ionisation n'est pas impactée par le feedback (voir section \ref{sec:pbfesc}), une étude similaire sera réalisée sur les flux radiatifs.
%Une question importante dans l’étude de réionisation est la quantité de rayonnement ionisant sortant des halos.
Nous avons vu que le rayonnement interagis les baryons, et nous verrons qu'il existe un liens entre la fraction baryonique déterminée plus haut (sec. \ref{sec:baryon_frac}) et les flux radiatif.


\subsubsection{Healpix}
\label{sec:healpix}

Nous cherchons à déterminer les flux à la surface des halos.
Par soucis de simplicité, nous utiliserons l'approximation du $R_{200}$.
Il faut donc discrétiser la sphère de rayon $R_{200}$ autour de chaque halo.
Pour se farie, cette discrétisation sera réalisée grâce à Healpix \citep{gorski_healpix:_2005}, un outil qui permets de répartir des points de manière uniforme sur la sphere (c.f. figure \ref{fig:HealPix}).
%L'association Healpix/grille sera réalisée par recherche de plus proche voisin 
Par la suite, tous les halos seront discrétisés en utilisant $3072$ points Healpix.
Il est probable, pour les halos les plus petits, qu'une cellule puisse être associée à plusieurs points Healpix.
%L'avantage de cette méthode est que toutes les cellules ont la même pondération.
Comme chaque point Healpix représente une portion constante de la surface de la sphère, la pondération en surface nécessaire pour la gestion des flux, se fait automatiquement.
Une fois la sphère générée, elle sera centrée sur la position du halo, et redimensionner de telle manière à ce que sa taille corresponde au $R_{200}$ du halo en question.
Puis, pour chaque point de la sphère Healpix, on effectuera une recherche de plus proche voisin à l'aide d'un KD-tree généré sur les centres de cellules, pour déterminer avec qu'elle cellule l'associer.
Nous avons donc accès aux cellules correspondant à la surface du halo.
%Nous connaissons maintenant l'ensemble des cellules au R200 de chaques halos.
Si il est possible de travailler directement avec les grandeurs scalaires, lors d'une analyse vectorielle, les vecteurs seront projetés sur la normale à la sphere.
%Mais nous cherchons a réaliser une étude sur les flux.


%Pour déterminer les flux autours de chaque halo, il est nécessaire de les englober d'une sphère.
%Cette sphère serra discrétisée a l'aide de Healpix (TODO ref) un outils qui permets de répartir des points de manière uniforme sur une surface sphérique .
\begin{figure}
	\centering
    \includegraphics[width=.45\linewidth]{img/03/healpix.jpg} 
    \caption[Sphère HealPix]{Exemple de sphère HealPix. Toutes les cellules sont uniformément réparties et de surfaces identiques.}
 	\label{fig:HealPix}
\end{figure}

\subsection{La vitesse du gaz au $R_{200}$}
\label{sec:snvitesses}
Dans le but de mettre en évidence l'efficacité des supernovae à expulser le gaz des halos, j'ai quantifié la vitesse moyenne du gaz autour des halos au niveau de leur $R_{200}$.
Pour chaque halos, la vitesse moyenne du gaz a sa surface est calculée.
Il s'agit des $3072$ vitesses obtenues avec Healpix, chacune projetée sur la surface à la sphère.
Par convention, la normale a la sphère pointe vers l'extérieur, les vitesses positives sont sortantes.
Donc si la vitesse moyenne est positive, le halo perd de la matière baryonique, et sinon il en accrète.
Les figures \ref{fig:R200speed1} et \ref{fig:R200speed2} représentes les résultats obtenus pour les quatre simulations de comparaison.

\begin{figure}
	\centering
	\includegraphics[height=.45\textheight]{img/03/flux_speed_noFEED.pdf} 
	\includegraphics[height=.45\textheight]{img/03/flux_speed_noSN.pdf} 
    \caption[Vitesse du gaz au $R_{200}$ 1]{Vitesse moyenne du gaz au $R_{200}$ en fonction de la masse du halo.
    Les tirets représentent la vitesse de chute libre et d'échappement.
    L'introduction de la radiation fait apparaître des flux sortants aux petites masses.
 	\label{fig:R200speed1}}
\end{figure}

\begin{figure}
	\centering
	\includegraphics[height=.45\textheight]{img/03/flux_speed_therm.pdf} 
	\includegraphics[height=.45\textheight]{img/03/flux_speed_kinetic.pdf} 
    \caption[Vitesse du gaz au $R_{200}$ 2]{Vitesse moyenne du gaz au $R_{200}$ en fonction de la masse du halo.
    Les tirets représentent la vitesse de chute libre et d'échappement.
    L'introduction des supernovæ génère des flux sortants dans une population de halo avec des étoiles veilles (ayant explosées en supernovae).
    Le feedback cinétique est capable de générer des flux sortants pour des halos plus massifs que dans le cas du feedback thermique.
 	\label{fig:R200speed2}}
\end{figure}

Dans le cas idéal, la matière chute sur le halo à la vitesse de chute libre.
A l'inverse, si de la matière est expulsé plus rapidement que cette vitesse limite, elle n'est plus gravitationnellement liée au halo.
Cette vitesse limite est représentée en tirets et s'exprime:
\begin{equation}
v_{lim} = \pm \sqrt{\frac{2 GM (r<R_{200})} {R_{200}} } 
\end{equation}

En absence de feedback, la quasi totalité des halos se trouvent sur cette relation limite.
De part la nature collisionnelle des baryons, cette vitesse est une limite supérieure, les baryons accrétés étant ralentit par la présence des baryons initialement présent au sein du halo.

Avec l'introduction de la radiation (figure \ref{fig:R200speed1}), il y a apparition d'une population de halos de faibles masses avec des vitesses moyenne positives.
Dans ce cas, la radiation est en mesure de générer des flux sortant, et donc de diminuer la fraction baryonique de cette classe de masse. $M< qq 10^8 M_\odot$.
Cette population ne possède pas d'étoile, ce qui suggère un effet extérieur au halo.
La vitesse d'expulsion moyenne est significativement plus faible que la vitesse d'échappement (1 ou 2 ordres de grandeurs).
Ces flux sortants observés peuvent expliquer la coupure de formation stellaire par le radiation pour les halos de moins $10^9 M_\odot$, observée dans des précédents travaux (eg \cite{ocvirk_cosmic_2015}).

De plus dans la partie accrétante du diagramme on observe que la vitesse limite est globalement plus faible (la population s'est légèrement écartée de la ligne limite).
Également, la dispersion des valeurs moyennes est plus grande, signifiant que la radiation n'agit pas de la même façon sur tout les halos.

Avec l'introduction des supernovae (figure \ref{fig:R200speed2}), il y a apparition d'une nouvelle population dans la partie haute du diagramme.
Cette population possède des étoiles aillant déjà explosées en supernovae, ce qui suggère un effet provenant de l'intérieur.
On observe que les halos possédant des vitesses positives peuvent avoir des masse plus élevées que dans le cas du simple feedback radiatif.
De plus, le feedback cinétique permet de générer des outflows sur des halos plus massif que dans le cas du feedback thermique.
Les vitesses moyennes sont proches de la vitesse d'échappement.

\subsection{Le flux radiatif au $R_{200}$}
\label{sec:radflow}

Comme dans ces simulations, la lumière est décrite comme une fluide (cf section \ref{sec:rad_solver}), un travail identique à celui de la section précédente peux être réalisé sur les flux radiatif.
La figure \ref{fig:R200rad} présente les résultats obtenus.
On observe que seul les halos avec des étoiles jeunes ont un flux de radiation sortant.
Que tous les halos de $M>2 \cdot 10^9 M_\odot$ ont un flux radiatif sortant et que le flux est d'autant plus important que ceux ci sont massifs.
Il n'y a pas de différences notables, dans la façon dont la radiation s'échappe des halos, entre les différents types de feedback, 
Cette observation est en accord avec les observations réalisées en section \ref{sec:pbfesc}, l'évolution de la fraction ionisée ne dépend pas du feedback car la quantité de photon s'échappant des halos ne dépend pas de la méthode de feedback de supernovae.

Les tirets représentes la masse d'un halo ayant un $R_200$ de taille équivalente à la résolution radiative.
En effet le transport des photons étant réalisé sur la grille de base (voir section \ref{sec:crta}) la résolution est plus faible que pour l'hydrodynamique.
% car le raffinement n'agit pas sur la radiation.
Il a déjà été observé qu'une partie des halos de moins de quelques $10^9 M_\odot$ voyaient leurs \ac{SFR} impacté par la radiation. %TODO ref
Malheureusement, cette limite en résolution correspond aussi à la taille limite des halos présentant un flux net entrant.
Des simulations à plus hautes résolution sont nécessaires pour déterminer si la limite en flux entrant n'est pas due à un effet de résolution.
Le fait que tous les halos ayant des étoiles jeunes aient un flux sortant et en faveur d'une résolution suffisante.% car si ce n'était pas le cas .

\begin{figure}
	\centering
	\includegraphics[height=.30\textheight]{img/03/flux_rad_noSN.pdf} 
	\includegraphics[height=.30\textheight]{img/03/flux_rad_therm.pdf} 
	\includegraphics[height=.30\textheight]{img/03/flux_rad_kinetic.pdf} 
    \caption[Flux de photons au $R_{200}$]{Flux de photons au total $R_{200}$ en fonction de la masse du halo. 
    Les halos les plus massifs ont un flux de photons sortant. 
    Une partie des halos les moins massifs ont un flux entrant.}
 	\label{fig:R200rad}
\end{figure}


\subsection{Budget de photons}

Connaissant les flux radiatif des halos, il est possible de déterminer la répartition du budget de photon en fonction de la masse des halos.
L'objectif est de déterminer quels sont les halos qui contribuent le plus à ioniser effectivement l'\ac{IGM} dans les simulations.
La figure \ref{fig:budget} présente les résultats obtenus en faisant la somme des flux sortant de tous les halos d'une certaine classe de masse.
Seuls les flux sortant sont considérés (et non les moyennes globales calculées en \ref{sec:radflow}).
Ce sont les halos de masses autour de $10^{10} M_\odot$ qui contribuent le plus au budget de photons global.

\begin{figure}
	\centering
	\includegraphics[width=.95\linewidth]{img/03/totout.pdf} 
    \caption[Budget de photon]{Budget de photons en fonction de la masse du halo et du modèle de supernovae. 
    Ce sont les halos de masses autour de $10^{10} M_\odot$ qui contribuent le plus au budget de photons global.
    Les tirets représentes la résolution radiative.
 	\label{fig:budget}}
\end{figure}


\subsection{Fraction d'échappement}
\label{sec:fesc}

En comparant la quantité de photon produite au sein du halo au flux de radiation sortants, il est possible de quantifier une grandeur centrale dans l'étude de la réionisation : la fraction d'échappement des photons $f_esc$.
La fraction d'échappement fait le lien entre ce qu'il est possible d'observer et la physique interne au halo.
%Le rapport de la quantité de photon produite sur la quantité de photon sortante, est la fraction d'échappement .
Nous avons vu que le taux de formation stellaire et donc le taux de production de photon, dépend du modèle de supernovae (vois section \ref{sec:sfr_halo}).
Mais nous avons vu que le flux de radiation sortant, lui ne dépend pas du modèle de supernovae.
Les $f_esc$ obtenues sont présentées sur la figure \ref{fig:fesc}.
Pour les halos en dessous de la limite en résolution (cf section \ref{sec:radflow}) l'interprétation est difficile, mais il semble que la $f_esc$ ne dépend pas du modèle de supernovae.
On y observe une nette augmentation de la fraction d'échappement pour les halos massifs, lors de l'introduction de supernovae cinétique.

Pour compenser le décalage entre le moment d'émmission des photons et le moment de leurs passage au $R_{200}$.
Les émissivités sont corrigées du temps nécessaire au parcours des photons entre le centre du halo et son $R_{200}$.



\begin{figure}
	\centering
	\includegraphics[width=.95\linewidth]{img/03/fesc.pdf} 
    \caption[Fraction d'échappement des photons]{Fraction d'échappement des photons en fonction de la masse du halo et du modèle de supernovae.
    Le modèle cinétique augmente significativement la fraction d'échappement des photons.
    Les tirets représentes la résolution radiative.
 	\label{fig:fesc}}
\end{figure}


\section{Conclusion}
%fonction de luminosité 

Nous avons comparer dans cette partie deux méthodes pour gérer les supernovae dans les simulations cosmologique.
L'objectif est de réaliser des simulations de type échantillons, dans le but de mieux cerner les processus actifs dans les simulations plus grandes.
Nous avons vu (sections \ref{sec:snmethod} et \ref{sec:snegy}) que la formation stellaire est fortement couplée à la méthode et à la quantité d'énergie injectée.
Une injection d'énergie sous forme cinétique sera plus efficace pour réguler la formation stellaire qu'une méthode par injection thermique, et plus la quantité d'énergie injectée sera élevée, meilleure sera la régulation.
La formation stellaire ne dépend que de la quantité de gaz présent dans la cellule (section \ref{sec:schmidt-kennicut}).
Sa régulation se fait par l'intermédiaire de l'expulsion des baryons des halos, qui est visible sur les cartes de champs (section \ref{sec:snmaps}), au sein des halos (section \ref{sec:baryon_frac}) et au niveau du $R_{200}$ (section \ref{sec:snvitesses}), privant ainsi les zone de formation stellaire de matière première.
Comme se sont les particules stellaires jeunes qui émettent du rayonnement ionisant (section \ref{sec:etoilerad}), la formation stellaire est directement liée a la quantité de radiation produite au sein des halos.
La production de photons est réduite en même temps que le \ac{SFR}.
En ayant moins de photons produit, le milieu devrait ioniser moins vite ce qui devrait être visible sur l'évolution de la fraction d'ionisation.
Or nous avons observé en section \ref{sec:pbfesc} que ce n'est pas le cas, l'histoire d'ionisation est presque indépendante de la méthode d'injection d'énergie de supernovae, alors que le budget de photon est impacté.
Nous avons expliqué cet effet en observant que les flux de radiation à la surface des halos ne semble pas impactés par le modèle d'injection d'énergie (section \ref{sec:radflow}).
Dans le cas du feedback cinétique, comme le flux sortant est inchangé alors que la production interne diminue, la fraction d'échappement augmente (section \ref{sec:fesc}).

Au final cette augmentation de la fraction d'échappement peut être expliquée par la diminution de la fraction baryonique.
En effet, comme la \ac{HMF} n'est pas impacté par le feedback (section \ref{sec:hmf}), la quantité de baryon dans les halos diminue avec l'introduction de supernovae.
Or, cette diminution de la quantité de baryon, permet à la radiation de s'échapper plus facilement.

Aux résolutions considérées dans cette étude, c'est a dire la résolution des plus grosses simulations de la réionisation de type CODA,  %TODO ref.
la compétition entre la diminution du taux de production de photon et l'augmentation de la fraction d'échappement semble neutre.
De plus, plus les supernovae sont puissante, plus le pas de temps est réduit, et plus le coût global de la simulation est important.
Ce qui fait que si une telle simulation est exécutée dans le seul intérêt d'étudier l'\ac{IGM}, l'introduction des supernovae n'est pas fondamentale.
A l'inverse, si L’intérêt est plus porté sur la morphologie des galaxies pendant la réionisation, il n'est plus possible de se passer de l'introduction du feedback.

%De plus nous observons que le budget de photons est principalement gouverné par les halos de masses intermédiaire autour de $10^{10} M_\odot$.








