\chapter{Introduction}

L'intégralité de l'astrophysique repose sur 3 piliers:
\begin{enumerate}
\item L'observation
\item La théorie
\item La simulation
\end{enumerate}

lien avec la méthode scientifique de manière générale. 
observation, modélisation et test de la théorie or en astro on ne peut pas tester directement donc on simule.

L'observation est le premier de ces pilier. 
Il est le plus ancien et celui sur lequel repose le plus de poids.
L'Homme a toujours regardé le ciel.
De plus il est commun a toute les disciplines scientifique.
Il n'est pas de science possible sans observation.
La principale difficulté ici, est que le point de vue que nous avons sur l'Univers est unique. 
Il nous est impossible de le regarder sous un autre angle.

Vient ensuite la théorie.
Lorsque l'on voit ces lumière sur la voûte celleste, nous sommes obligé de nous poser la question essentielle de leur provenance.
Cette question mène a l'élaboration de diverse formulation tentant d'expliquer comment (et pourquoi) le ciel s'illumine la nuit.  
Dans le cadre de l'étude de l'univers dans sont ensemble, cette théorie est nommée cosmologie et repose sur des concept mathématiques complexes
Il nous est impossible de d'effectuer des expériences sur l'univers, notre porté d'interaction est bien trop réduite.

Enfin, le dernier pilier : la simulation.
Pilier le plus récent il est sensé palier au problème des deux autres : l'impossibilité de changer de point de vue ou de tester les théories élaborées.
Ici sous entendue la simulation numérique, il est beaucoup plus récente et dépend grandement de la technologie.
C'est celui vers lequel j'ai choisis de me diriger.


Notre compréhension actuelle de l'univers s'inscrit dans le cadre du modèle standard de la cosmologie.
Ce modèle repose sur un univers non statique et en expansion, aillant une origine, le Bigbang.
A une époque, l'univers était extrêmement chaud et concentré, il n'y avait alors ni étoiles ni galaxies.
Mais en grandissant,  le plasma primordial c'est refroidit, les premiers atomes se sont formé lors de se que l'on nomme la recombinaison, qui a mené a l'émission du fond diffus cosmologique.
L'univers était alors extrêmement homogène que ce soit en densité ou en température, mais cette homogénéité n'était pas parfaite.
S'en suit une période ou les in-homogénéités primordiales se sont effondrées sur elles même du fait de la gravité.
Puis ces in-hommogeneites sont devenues suffisamment dense pour former les première étoiles.
Le matériaux disponible pour leurs formations était alors abondant.
On pense que ces étoiles étaient beaucoup plus massives, et donc beaucoup plus énergétique que les étoiles observées actuellement.
Ces étoiles ont émis un rayonnement suffisamment énergétique pour séparer les électrons et les protons lies au moment de la recombinaison.
L'univers c'est alors retrouvé une nouvelle fois dans un état majoritairement ionisé. 
C'est cette transition entre un univers neutre et froid vers un univers chaud et ionisé que l'on appel l'époque de reionisation.
L'apparition des première sources lumineuse a eu un impact sur la façon dont la matière c'est organisé pour former les galaxies.
Il est probable que l'univers que l'on observe aujourd'hui, ai conservé les traces de cette grande époque de transition.

Une des difficulté a l'étude de la reionisation est l'époque a laquelle elle s'est produite, on considère actuellement qu'elle a eu lieu dans le premier milliard d'année de l'univers.
Or, pour observer l'univers jeune, il faut regarder loin, tellement loin que les meilleur télescopes actuel sont tut juste assez performant pour atteindre des époques aussi lointaines.
Il faudra attendre encore au moins un décennie avant la mise en place des prochaine générations de télescope assez puissant pour observer les environnement de formation de ces première sources lumineuses.

Mais en attendant, si les possibilité d'observation sont restreinte, nous pouvons utiliser d'autre méthodes pour tenter de comprendre les phénomènes en cours a cette époque.
Nous pouvons utiliser les simulations numériques.
En effet, les phénomènes en action pendant la reionisation sont nombreux et les modèles analytique trouvent leurs limites.
Avec l'avancée exponentielle des capacités de calculs, les ordinateurs se transforme petits a petit en véritable laboratoire pour les astrophysiciens.
A l'heure actuelle il commence a être possible de simuler l'effondrement de structures cosmologique, contenant du gaz, formant des étoiles qui émettent du rayonnement 

Ces simulations ont pour objectifs d'aider a comprendre les grandes questions en suspend dans l'étude de la reionisation.
Voici un apercu de ces questions : 

\begin{itemize}
\item Quand sont apparue les premières sources lumineuses?
Nous verrons que les observations commencent a imposer certaines contraintes sur la fin de la reionization mais  nous n'avons actuellement qu'une vague idée de la durée du processus.

\item L'univers a t il été réionisé par quelques grosses sources très lumineuses ou par de nombreuses source moins énergétique.
La question reste ouverte de savoir si ce sont les quasars, sources relativement rare mais pouvant etre extrement energétique, ou les galaxies plus modeste mais beaucoup plus nombreuse.
Dans le cas ou lce serai les galaxies, serai ce les les plus légéres, extrement nombreuses ou les plus massives.

\item Comment ces premières génération d'étoiles ont influencé m'apparitions des suivante, et ont elles laissées des traces encores visible dans l'environement proches?

\end{itemize} 

En repondant a certaine de ces questions, les simulations numeriques prépare les futures mission d'observation.
En etudiant au prealable ce que l'on cherche a  observer, on a plus de chance d'observer au bon endroit et de la bonne facon.


Au stade actuel de notre compréhension de l'univers, les simulations numériques ont a la fois de très belles réussites mais souffres également de 

\subsection*{La réionisation}
Le dernier grand changement que l'univers a subit date de son premier milliard d'années.
L'apparition des premières étoiles a transformé l'Univers alors froid et sombre en un univers chaud et inondé de lumière.
Cette transition est due a l'effondrement gravitationnel du gaz qui a permit, par endroits, une élévation de densité et de température suffisante pour réamorcer des réactions de fusion thermonucléaire.
Il s'agit de la première génération d'étoiles.
Le matériaux disponible pour leurs formations était alors abondant.
On pense que ces étoiles étaient beaucoup plus massives, et donc beaucoup plus énergétique que les étoiles observées actuellement.
Cette première génération à émis un fort rayonnement ultraviolet qui a grandement impacté le milieu environnant, en le chauffant par effet thermique et en le déplacant par effet de pression de radiation.
De plus, a la fin de leur vie, ces étoiles massives ont explosées en supernovae, effectuant alors un puissant chauffage ainsi qu'un fort brassage du gaz.
En changeant la configuration du milieu, ces premières étoiles ont modelées les lieux d'apparition des générations suivantes et donc la distribution de matière observée aujourd'hui.
La vitesse de la lumière étant finie, il fallut un certain temps pour celle ci puisse atteindre tous les recoins de l'univers. Il est estimé aujourd'hui que les premières étoiles sont apparues alors que l'univers était âgé d'environ 300 m
illions d'années. Il fallut alors 700 millions d'années supplémentaires pour que le rayonnement atteigne tout ses recoins, situant donc la fin de la période de  réionisation à milliard d'années après le Big Bang.

\subsection*{Les simulation numeriques}
Pour étudier des phénomènes aussi fortement couplés que ceux considérés dans le cas de l'époque de réionisation, il est nécessaire d'avoir recours à des simulations numérique. 
Le but de ces simulations est de reproduire les observations sur la distribution de matière dans l'Univers.

Les premières simulations cosmologiques ne considéraient l'évolution que de la composante non collisionnelle de la matière, ie la matière noire.
Comme la matière noire constituent la masse la plus abondante de l'Univers, ces simulation permettent de suivre l'évolution de la distribution de matière sur les grandes échelles. 
Mais par essence la matière noire est une matière invisible. 
Il manquait donc une compossante importante : la matière visible, les barrions.

Le calcul de l'hydrodynamique du gaz fut alors introduit, puis avec lui les premiers modèles de formation stellaire apparurent.
Mais la communauté a vite été confronté a un important problème: le gaz refroidissait trop.
Ce qui avait pour conséquence qu'il s'effondrait sur lui même trop rapidement et formait trop d'étoiles.
Pour palier a ce problème, il fut proposé d'injecter de l'énergie dans les endroits les plus dense.
Cette énergie, introduite par les supernovae, permet de chauffer le gaz et ralentis son effondrement. 

%Depuis très récemment, une troisième physique est devenue  dans les simulations l'influence de la radiation sur le milieu est également prise en compte.
Aujourd'hui, l'intérêt est porté sur l'introduction d'une nouvelle physique, celle du rayonnement.
Le rayonnement émit par les étoiles, va changer les propriétés physico-chimique et thermique du gaz qui les environnent et ainsi modeler les lieux d\'apparitions des générations future d'étoiles


\subsection*{Le groupe local}
Le groupe local est un ensemble de quelques dizaines de galaxies, dont les principales représentantes sont la Voie Lactée et notre voisine Andromède.
Il s'agit de notre environnement galactique proche.
Ce qui le rend facilement observable.
L'observation de cet environnement nous fournis des informations essentielles sur la cosmologie de l'Univers.

Plus la precision des observations augmente, plus il est nécessaire d'avoir des simulations résolues disposant de toutes la physique nécessaire pur expliquer globalement les échelles considérées.
Les premières simulations de matière noires ont permit d'expliquer les observations réalisées sur la distribution de la matière aux grandes échelles.
Lorsque les capacité de calcul se sont révélées suffisantes pour explorer des résolutions plus fines, il est vite apparu un certain décalage entre simulation et observation. 
L'introduction de l'hydrodynamique a permis d'augmenter l'accord entre les deux, jusqu'à un second palier de résolution.
Aujourd'hui, l'introduction de la physique du rayonnement va certainement permette de diminuer encore les échelles auxquelles les simulations sont sont en accord avec les observations.
L'ordre de grandeur de ces échelles est celui de la taille du groupe local.

De plus, les échelles que l'on considère dans les simulations cosmologiques a rayonnement couplé se rapproche de plus en plus des échelles considérés dans les simulations d'évolution de galaxies.
L'objectif est ici de faire le lien entre la physique de l'univers dans son ensemble (la cosmologie) et de la physique régissant l'évolution de notre galaxie ou de sa voisine (la physique galactique).



\section*{Travaux}

Je travail avec Dominique Aubert au développement d'un code nommé \emma, capable de simuler l'évolution de la matière noire, du gaz et de la radiation.
Il a pour principal objectif l'étude de la période de réionisation.
\emma\ est un code a maille adaptative (AMR) qui a la particularité d'être massivement parallèle et d'être accéléré par processeurs graphiques.

Ma tache principale a été jusqu'ici d'implémenter un scénario de formation stellaire, ainsi qu'un modèle de feedback de supernovae.
Mais d'une manière plus générale j'ai contribué à divers aspect du code, comme la gestion des paramètres utilisateur, l'écriture des données de sorties ou encore la documentation.
Je développe également une librairie d'analyse des fichiers générés par \emma.
J'ai également contribué au développement d'un code de visualisation et d'exploration de simulations astrophysique.

\subsection*{Mod\`ele de formation stellaire}
L'étude du rayonnement émis dans l'Univers, passe immanquablement par l'étude de la formation stellaire.
J'ai implémenté dans \emma, un modèle de transformation du gaz en particule stellaire.
Ce modèle est basé sur un critère de densité.
Lorsque, sous l'effet de la gravitation, une cellule devient suffisamment dense, celle ci est marquée comme étant autorisée a former des étoiles.
Ensuite, toute les cellules autorisées vont former une certaine quantité d'étoiles, fonction de l'état local de la cellule, et suivant une loi empirique issue de l'observation : la loi de Schmidt-Kennicutt.

Cette recette nécessite l'introduction de plusieurs paramètres libres.
Une partie importante du temps a été consacrée à la calibration de ces paramètres.

\subsection*{Mod\`ele de supernovae}

Historiquement, les modèles de formation stellaire se sont rapidement confronté à un problème majeur : ils n'arrivaient pas à reproduire les observations en terme de quantité d'étoiles crées.
Une réponse à ce problème fut l'introduction des supernovaes. 
Les explosions de supernovaes injectent une quantité non négligeable d'énergie dans le milieu environnant.
Cette énergie supplémentaire perturbe le milieu et régule la formation stellaire.

J'ai implementé un modèle d'injection d'énergie dans le solveur hydrodynamique d'\emma.
Ce modèle considère deux types d'énergie. 
La première, thermique, va dissiper une certaine partie de l'énergie disponible dans le chauffage du gaz.
La seconde, cinétique, va mettre en mouvement le gaz environnant avec le reste de l'énergie.
Une certaine proportion de la masse de la particule stellaire est retournée dans le milieu. 
L'énergie totale et la masse des éjectas étant contrainte par un autre modèle, Starburst99.

De la même manière que précédemment, l'implémentation de ce modèle a introduit plusieurs paramètres libres qu'ils a fallut calibrer.





