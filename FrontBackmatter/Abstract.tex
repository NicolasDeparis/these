%*******************************************************
% Abstract
%*******************************************************
%\renewcommand{\abstractname}{Abstract}
\pdfbookmark[1]{Résumé}{Résumé}
\begingroup
\let\clearpage\relax
\let\cleardoublepage\relax
\let\cleardoublepage\relax

\chapter*{Résumé}
L’époque de réionisation (EoR) est une phase de grands changements qu’a subit l’Univers dans son premier milliard d’années. Suite à l’apparition des premières étoiles et à l’émission de photons énergétique par ces dernières, l’hydrogène a été réionisé. Cette transition à eu un impact sur la formation des galaxies. 

J’ai activement participé au développement d’EMMA, un code de simulation numérique aillant pour objectif d’étudier les processus a l’œuvre durant l’EoR. J’ai développé et implémenté un modèle de formation et d’évolution stellaire. Ces travaux ont contribué à la réalisation d’une simulation dédiée a l’étude de l’EoR parmi les plus grosses réalisées a l’heure actuelle. J’ai contribué au développement d’outils dédiés a l’exploration de simulations de ce type.

J’ai étudié la façon dont le rayonnement s’échappe des galaxies en fonction des paramètres du modèle stellaire, et montré que les supernovae peuvent augmenter la fraction de photons libérés.

J’ai également étudié la propagation des fronts d’ionisation et montré qu’il était possible de réduire la vitesse de la lumière par 3 (et ainsi diminuer le temps de calcul du transfert du rayonnement par 3), tout en conservant des résultats corrects.

\vspace{0.5cm}

Mots clefs : Cosmologie, age sombres, réionisation, premières étoiles, méthodes numériques.
\vfill

\newpage

\begin{otherlanguage}{english}
\pdfbookmark[1]{Abstract}{Abstract}
\chapter*{Abstract}
The epoch of reionization (EoR) is a phase of big changes in the first billion years of the Universe history. After the apparition of the first stars and the emission of energetic radiation by thoses ones, the hydrogen was reionized. This transition has an impact on the galaxies formations.

I was part of the development team of EMMA, a numerical simulation code who aimed to study the  processes happening during the EoR. I developed and implement a stellar formation and evolution model. These works contributed to the realisation of one of the biggest simulation dedicated to the study of the EoR yet. I contribute to the development of a tool dedicated to the exploration of this kind of simulations.

I study how the radiation escaped the galaxies as a function of the parameters of the stellar model, and showed that supernovae could increase the ratio of escaping photon.

I also studied the ionization fronts propagation and showed that the speed of light could be reduced by a factor 3 (and then divide the computational cost of the radiative transfer by 3), while keeping corrects results.

\vspace{0.5cm}

Keywords : Cosmology, dark ages, reionization, first stars, numerical methods
\end{otherlanguage}

\endgroup			

\vfill




