%*******************************************************
% Abstract
%*******************************************************
%\renewcommand{\abstractname}{Abstract}
\pdfbookmark[1]{Résumé}{Résumé}
\begingroup
\let\clearpage\relax
\let\cleardoublepage\relax
\let\cleardoublepage\relax

\chapter*{Résumé}
L’époque de réionisation (EoR) est une phase de grands changements qu’a subi l’Univers dans son premier milliard d’années. 
Suite à l’apparition des premières sources de rayonnement et à l’émission de photons énergétiques par ces dernières, l’hydrogène a été réionisé. 
Cette transition à eu un impact sur la formation des galaxies et leur contenance stellaire.

J’ai activement participé au développement d’EMMA, un code de simulation numérique aillant pour objectif d’étudier les processus a l’œuvre durant l’EoR. 
J’y ai développé et implémenté un modèle de formation et d’évolution stellaire et ces travaux ont contribué à la réalisation de "CoDa I AMR" une simulation dédiée à l’étude de l’EoR parmi les plus grandes réalisées à l’heure actuelle. 
J’ai également contribué au développement d’outils dédiés à l’exploration de simulations de ce type.

J’ai étudié la façon dont le rayonnement s’échappe des galaxies en fonction des paramètres du modèle stellaire, et montré qu'aux résolutions d’intérêt les supernovae peuvent augmenter la fraction de photons libérés.

J’ai également étudié la propagation des fronts d’ionisation et montré qu’il était possible de réduire la vitesse de la lumière par 3 (et ainsi diminuer le temps de calcul du transfert du rayonnement par 3), tout en conservant des résultats corrects.
CoDa I AMR a permis d'étudier le lien entre l'histoire de réionisation des halos et leurs masses actuelles.
Cette étude tend à confirmer l'hypothèse d'une réionisation précoce et interne du Groupe Local.


\vspace{0.5cm}

Mots clefs : Cosmologie, age sombres, réionisation, premières étoiles, méthodes numériques.
\vfill

\newpage

\begin{otherlanguage}{english}
\pdfbookmark[1]{Abstract}{Abstract}
\chapter*{Abstract}
The epoch of reionization (EoR) is a great transition in the first billion years of the Universe history. 
After the apparition of the first sources and the emission of energetic radiation by thoses ones, the hydrogen was reionized. 
This transition has an impact on the galaxies formations and their stellar content.

I was part of the development team of EMMA, a numerical simulation code who aimed to study the  processes happening during the EoR. 
I developed and implemented a stellar formation and evolution model. 
These works contributed to the realisation of "CoDa I AMR" one of the largest simulation dedicated to the study of the EoR yet. 
I contribute to the development of a tool dedicated to the exploration of this kind of simulations.

I studied how the radiation escaped the galaxies as a function of the parameters of the stellar model, and showed that supernovae could increase the ratio of escaping photon at our resolutions.

I also studied the ionization fronts propagation and showed that the speed of light could be reduced by a factor 3 (and then divide the computational cost of the radiative transfer by 3), while keeping corrects results.
CoDa I AMR was used to study the link between reionization histories and presents halos masses.
This study tends to confirm the hypothesis of an early and internal reionization of the Local Group.

\vspace{0.5cm}

Keywords : Cosmology, dark ages, reionization, first stars, numerical methods
\end{otherlanguage}

\endgroup			

\vfill




